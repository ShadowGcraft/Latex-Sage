% !TEX parameter = xelatex --shell-escape -synctex=1 -interaction=nonstopmode

%\documentclass[a4paper,12pt]{smfart}

%\documentclass[a4paper,11pt]{book}
%\documentclass[a4paper,landscape,20pt]{extarticle} %% landscape ou bien portrait (par défault)
\documentclass[a4paper,14pt]{extarticle} %% on peut utiliser les tailles: 8pt, 9pt,14pt, 17pt et 20pt
%\documentclass[a4paper,11pt]{report}
\usepackage[french]{babel}
\usepackage[T1]{fontenc}
\usepackage[utf8]{inputenc}

%\usepackage{amsmath,amssymb,amsfonts,url,xspace,smfthm}
\usepackage{amsmath,amsthm,amssymb,amsfonts,url,xspace}
\usepackage[mathscr]{eucal}

%\usepackage{exscale}

\usepackage{diagbox}

\usepackage{fancybox}

\usepackage{xcolor}

%%%%%%%%%%%%%%%%%%%%%%
% - les theoremes
\newtheorem{definition}{D\'efinition}
\newtheorem{theoreme}{Th\'eor\`eme}
\newtheorem{proposition}{Proposition}
\newtheorem{lemme}{Lemme}
\newtheorem{corollaire}{Corollaire}
\newtheorem{propriete}{Propri\'et\'e}
\newtheorem{axiome}{Axiome}
\newcommand{\initialisecompteurs}{
       \setcounter{definition}{0}%
       \setcounter{theoreme}{0}%
       \setcounter{proposition}{0}%
       \setcounter{lemme}{0}%
       \setcounter{corollaire}{0}%
       \setcounter{propriete}{0}%
       \setcounter{axiome}{0}%
       \setcounter{section}{0}%
       \setcounter{subsection}{0}%
       \setcounter{subsubsection}{0}%
}


\usepackage[left=2.cm,right=2.cm,top=2.cm,bottom=2cm,bindingoffset=0cm, headsep=0cm,headheight=0cm]{geometry}

%\usepackage{qrcode}
%\usepackage[draft,forget]{qrcode}
%\usepackage{rmsmacros} 
\usepackage{graphicx}
%\usepackage{hyperref}
%\usepackage[hidelinks]{hyperref}
%\usepackage{tabvar}
%\usepackage[tight]{shorttoc}
%\newcommand{\sommaire}{\shorttoc{Sommaire}{1}}

\graphicspath{ {./images/} }
\everymath{\displaystyle}
\usepackage{fontspec}  % fontspec et xunicode sont facultatifs
\setmainfont[Ligatures=TeX]{Helvetica}   % Police romaine, utilisée dans le corps du document
%\setmainfont[Ligatures=TeX]{Luciole}   % Police romaine, utilisée dans le corps du document



\def\eclaire{\mathbb}

\def\R{\ensuremath{\eclaire R}}
\def\N{\ensuremath{\eclaire N}}
\def\C{\ensuremath{\eclaire C}}
\def\Z{\ensuremath{\eclaire Z}}
\def\Q{\ensuremath{\eclaire Q}}

%\newcommand{\sh}{\mathop{\mathrm{sh}}\nolimits}
\renewcommand{\sinh}{\mathop{\mathrm{sh}}}
\renewcommand{\cosh}{\mathop{\mathrm{ch}}}
\renewcommand{\tanh}{\mathop{\mathrm{th}}}
\renewcommand{\arcsin}{\mathop{\mathrm{Arc\,sin}}}
\renewcommand{\arccos}{\mathop{\mathrm{Arc\,cos}}}
\renewcommand{\arctan}{\mathop{\mathrm{Arc\,tan}}}
\renewcommand{\Im}{\mathop{\mathfrak{I}\mathrm{m}}}
\renewcommand{\Re}{\mathop{\mathfrak{R}\mathrm{m}}}

 


\DeclareMathOperator{\sh}{sh}
\DeclareMathOperator{\ch}{ch}


\renewcommand{\baselinestretch}{1.5}

\begin{document} 



\setlength{\parindent}{0pt}


%\bf \Huge

\section*{Notations.}

On note $\wedge$ le pgcd et $\vee$ le ppcm, par ailleurs on préfère la notation $a\equiv b \pmod{n}$ pour exprimer que $a$ est congru à $b$ modulo $n$.

\section*{Exercice~1.}

Soit $n\geqslant 2$. Calculer :
\begin{enumerate}
\item $n \wedge (2n+1)$
\item $n \vee (2n+1)$
\item $(n-1)\wedge(2n+1)$
\item $(n-1)\vee(2n+1)$
\end{enumerate}

\subsection*{Ma solution}

\subsubsection*{$n \wedge (2n+1)$ ?}

Comme le reste de la division euclidienne entre les deux nombres vaut $1$, alors  $n \wedge (2n+1)$ = 1



\subsubsection*{$n \vee (2n+1)$ ?}

Comme $n \wedge (2n+1) = 1$ alors $n \vee (2n+1) = n \times (2n+1)$



\subsubsection*{$(n-1)\wedge(2n+1)$ ?}

La division euclidienne de $(2n+1)$ par $(n-1)$ vaut 3. Donc $(n-1)\wedge(2n+1) = 3$



\subsubsection*{$(n-1)\vee(2n+1)$ ?}

On a trois cas qui se présente à nous. \\

\begin{enumerate}
\item  Lorsque $n \equiv 1 \pmod{3}$, on a $(n-1)$ et $(2n+1) \equiv 0 \pmod{3}$. \\ Donc $(n-1)\vee(2n+1)$ vaut $(n-1) \times (2n+1)$
\item Lorsque $n \equiv 2 \pmod{3}$, on a $(n-1) \equiv 1 \pmod{3}$ et $(2n+1) \equiv 2 \pmod{3}$. \\ Donc $(n-1)\vee(2n+1)$ vaut $\frac{(n-1) \times (2n+1)}{3}$
\item Lorsque $n \equiv 0 \pmod{3}$, on a $(n-1) \equiv 2 \pmod{3}$ et $(2n+1) \equiv 1 \pmod{3}$. \\ Donc $(n-1)\vee(2n+1)$ vaut $\frac{(n-1) \times (2n+1)}{3}$
\end{enumerate}


\subsection*{Solution, proposée par le manuel, de l'exercice~1.}
\begin{enumerate}

\item $n \wedge (2n+1)$ ? \\
La division euclidienne de $2n+1$ par $n$ s'exprime par l'égalité $2n+1= 2\times n +1$, c'est-à-dire $2n+1-2n=1$ d'où on conclut que les entiers $(2n+1)$ et $n$ sont premiers entre eux.

\item $n \vee (2n+1)$ ?\\
Comme le pgcd de $(2n+1)$ et $n$ vaut $1$, alors le ppcm de $(2n+1)$ et $n$ est le produit $(2n+1) \times n$.

\item $(n-1)\wedge(2n+1)$ ? \\
La division euclidienne de $2n+1$ par $n-1$ s'exprime par l'égalité $2n+1= 2\times (n-1) +3 $,  d'où on conclut que le pgcd de $(n-1)$ et $(2n+1)$ est un diviseur de $3$, donc est égal à $3$ ou bien $1$.

\begin{itemize}

\item Dans le cas où $n \not\equiv 1 \pmod{3}$ 
 implique $n-1 \not\equiv 0 \pmod{3}$ c'est-à-dire $n-1$ n'est pas divisible par $3$  et donc $(n-1)\wedge(2n+1)=1$.
 
\item Dans le cas où $n \equiv 1 \pmod{3}$, on a alors $2n+1\equiv 2\times 1+1 \equiv 3  \pmod{3}$ c'est-à-dire $2n+1\equiv 0 \pmod{3}$, donc 3 divise $2n+1$.\\
$n \equiv 1 \pmod{3}$ implique $n-1 \equiv 0 \pmod{3}$ c'est-à-dire 3 divise $n-1$  et donc $(n-1)\wedge(2n+1)=3$.
\end{itemize}


\item $(n-1)\vee(2n+1)$ ?\\
Les calculs des pgcd ci-dessus permettent de trouver aisément les ppcm. En conclusion on a :
\begin{description}
\item[$\bullet$] si $n \equiv 1 \pmod{3}$, alors $(n-1)\vee(2n+1) = \frac{(n-1)(2n+1)}{3} $;
\item[$\bullet$] si $n \not\equiv 1 \pmod{3}$, alors $(n-1)\vee(2n+1) = (n-1)(2n+1)$. 
\end{description}


\end{enumerate}

\section*{Exercice~2.}

Soit $(a,b,c) \in (\N*)^3$ tel que $a^2+b^2=c^2$ et $ a \wedge b \wedge c =1$.\\ Montrer que $a \wedge b = a \wedge c = b \wedge =1$.

\subsection*{Solution de l'exercice~2.}



\section*{Exercice~3.}

Soit $(a,b,c) \in (\N*)^3$ tel que $a^2+b^2=c^2$ et $ a \wedge b =1$.\\ Montrer que $a$ et $ b$ ne sont pas de même parité.\\
\textsl{Indication. On pourra utiliser des congruences modulo 4.}


\subsection*{Solution de l'exercice~3.}
















\end{document}