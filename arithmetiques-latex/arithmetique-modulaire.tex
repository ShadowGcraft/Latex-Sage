% !TEX parameter = xelatex --shell-escape -synctex=1 -interaction=nonstopmode

%\documentclass[a4paper,12pt]{smfart}

%\documentclass[a4paper,11pt]{book}
%\documentclass[a4paper,landscape,20pt]{extarticle} %% landscape ou bien portrait (par défault)
\documentclass[a4paper,14pt]{extarticle} %% on peut utiliser les tailles: 8pt, 9pt,14pt, 17pt et 20pt
%\documentclass[a4paper,11pt]{report}
\usepackage[french]{babel}
\usepackage[T1]{fontenc}
\usepackage[utf8]{inputenc}

%\usepackage{amsmath,amssymb,amsfonts,url,xspace,smfthm}
\usepackage{amsmath,amsthm,amssymb,amsfonts,url,xspace}
\usepackage[mathscr]{eucal}

%\usepackage{exscale}

\usepackage{diagbox}

\usepackage{fancybox}

\usepackage{xcolor}

%%%%%%%%%%%%%%%%%%%%%%
% - les theoremes
\newtheorem{definition}{D\'efinition}
\newtheorem{theoreme}{Th\'eor\`eme}
\newtheorem{proposition}{Proposition}
\newtheorem{lemme}{Lemme}
\newtheorem{corollaire}{Corollaire}
\newtheorem{propriete}{Propri\'et\'e}
\newtheorem{axiome}{Axiome}
\newcommand{\initialisecompteurs}{
       \setcounter{definition}{0}%
       \setcounter{theoreme}{0}%
       \setcounter{proposition}{0}%
       \setcounter{lemme}{0}%
       \setcounter{corollaire}{0}%
       \setcounter{propriete}{0}%
       \setcounter{axiome}{0}%
       \setcounter{section}{0}%
       \setcounter{subsection}{0}%
       \setcounter{subsubsection}{0}%
}


\usepackage[left=2.cm,right=2.cm,top=2.cm,bottom=2cm,bindingoffset=0cm, headsep=0cm,headheight=0cm]{geometry}

%\usepackage{qrcode}
%\usepackage[draft,forget]{qrcode}
%\usepackage{rmsmacros} 
\usepackage{graphicx}
%\usepackage{hyperref}
%\usepackage[hidelinks]{hyperref}
%\usepackage{tabvar}
%\usepackage[tight]{shorttoc}
%\newcommand{\sommaire}{\shorttoc{Sommaire}{1}}

\graphicspath{ {./images/} }
\everymath{\displaystyle}
\usepackage{fontspec}  % fontspec et xunicode sont facultatifs
\setmainfont[Ligatures=TeX]{Helvetica}   % Police romaine, utilisée dans le corps du document
%\setmainfont[Ligatures=TeX]{Luciole}   % Police romaine, utilisée dans le corps du document



\def\eclaire{\mathbb}

\def\R{\ensuremath{\eclaire R}}
\def\N{\ensuremath{\eclaire N}}
\def\C{\ensuremath{\eclaire C}}
\def\Z{\ensuremath{\eclaire Z}}
\def\Q{\ensuremath{\eclaire Q}}

%\newcommand{\sh}{\mathop{\mathrm{sh}}\nolimits}
\renewcommand{\sinh}{\mathop{\mathrm{sh}}}
\renewcommand{\cosh}{\mathop{\mathrm{ch}}}
\renewcommand{\tanh}{\mathop{\mathrm{th}}}
\renewcommand{\arcsin}{\mathop{\mathrm{Arc\,sin}}}
\renewcommand{\arccos}{\mathop{\mathrm{Arc\,cos}}}
\renewcommand{\arctan}{\mathop{\mathrm{Arc\,tan}}}
\renewcommand{\Im}{\mathop{\mathfrak{I}\mathrm{m}}}
\renewcommand{\Re}{\mathop{\mathfrak{R}\mathrm{m}}}




\DeclareMathOperator{\sh}{sh}
\DeclareMathOperator{\ch}{ch}


\renewcommand{\baselinestretch}{1.5}

\begin{document} 

\pagecolor{gray}

\setlength{\parindent}{0pt}


%\bf \Huge

\textcolor{white}{\section*{Exercice~1}}

\textcolor{white}{Soit $n\geqslant 2$. Calculer :
\begin{enumerate}
\item $n \vee (2n+1)$
\item $n \wedge (2n+1)$
\item $(n-1)\vee(2n+1)$
\item $(n-1)\wedge(2n+1)$
\end{enumerate}
}

\textcolor{white}{\subsection*{Solution de l'exercice~1}}

\textcolor{white}{Quelques explications pour écrire les textes d'arithmétique modulaire.}

\begin{tabular}{lll}
\verb|$a\equiv b \pmod{n}$| & $a\equiv b \pmod{n}$ \\
\verb|$a\equiv b \mod{n}$|  & $a\equiv b \mod{n}$ \\
\verb|$a\equiv b \pod{n}$|  & $a\equiv b \pod{n}$ \\
\verb|$a\equiv b \bmod{n}$| & $a\equiv b \bmod{n}$ & (wrong) 
\end{tabular}


\textcolor{white}{\section*{Exercice~2}}

\textcolor{white}{Soit $(a,b,c) \in (\N*)^3$ tel que $a^2+b^2=c^2$ et $ a \wedge b \wedge c =1$.\\ Montrer que $a \wedge b = a \wedge c = b \wedge =1$.}

\textcolor{white}{\subsection*{Solution de l'exercice~2}}



\textcolor{white}{\section*{Exercice~3}}

\textcolor{white}{Soit $(a,b,c) \in (\N*)^3$ tel que $a^2+b^2=c^2$ et $ a \wedge b =1$.\\ Montrer que $a$ et $ b$ ne sont pas de même parité.\\
\textsl{Indication. On pourra utiliser des congruences modulo 4.}}


\textcolor{white}{\subsection*{Solution de l'exercice~3}}
















\end{document}