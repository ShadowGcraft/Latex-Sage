% !TEX TS-program = sage 
% !TEX TS-program = xelatex
% !TEX encoding = UTF-8 Unicode
% The following lines are standard LaTeX preamble statements.

%\documentclass[a4paper,12pt]{smfart}

%\documentclass[a4paper,12pt]{smfart}

%\documentclass[a4paper,11pt]{book}
%\documentclass[a4paper,landscape,20pt]{extarticle} %% landscape ou bien portrait (par défault)

\documentclass[a4paper,14pt]{extarticle} %% on peut utiliser les tailles: 8pt, 9pt,14pt, 17pt et 20pt

%\documentclass[a4paper,11pt]{report}
%\documentclass[a4paper,14pt]{extreport} %% on peut utiliser les tailles: 8pt, 9pt,14pt, 17pt et 20pt


\usepackage[french]{babel}
\usepackage[T1]{fontenc}
\usepackage[utf8]{inputenc}

%\usepackage{amsmath,amssymb,amsfonts,url,xspace,smfthm}
\usepackage{amsmath,amsthm,amssymb,amsfonts,url,xspace}
\usepackage[mathscr]{eucal}

%\usepackage{exscale}

\usepackage{diagbox}

\usepackage{fancybox}

\usepackage{xcolor}

%%%%%%%%%%%%%%%%%%%%%%
% - les theoremes
\newtheorem{definition}{D\'efinition}
\newtheorem{theoreme}{Th\'eor\`eme}
\newtheorem{proposition}{Proposition}
\newtheorem{lemme}{Lemme}
\newtheorem{corollaire}{Corollaire}
\newtheorem{propriete}{Propri\'et\'e}
\newtheorem{axiome}{Axiome}
\newcommand{\initialisecompteurs}{
       \setcounter{definition}{0}%
       \setcounter{theoreme}{0}%
       \setcounter{proposition}{0}%
       \setcounter{lemme}{0}%
       \setcounter{corollaire}{0}%
       \setcounter{propriete}{0}%
       \setcounter{axiome}{0}%
       \setcounter{section}{0}%
       \setcounter{subsection}{0}%
       \setcounter{subsubsection}{0}%
}


\usepackage[left=2.cm,right=2.cm,top=2.cm,bottom=2cm,bindingoffset=0cm, headsep=0cm,headheight=0cm]{geometry}

%\usepackage{qrcode}
%\usepackage[draft,forget]{qrcode}
%\usepackage{rmsmacros} 
\usepackage{graphicx}
%\usepackage{hyperref}
%\usepackage[hidelinks]{hyperref}
%\usepackage{tabvar}
\usepackage[tight]{shorttoc}
\newcommand{\sommaire}{\shorttoc{Sommaire}{1}}

\graphicspath{ {./images/} }
\everymath{\displaystyle}
%\usepackage{fontspec}  % fontspec et xunicode sont facultatifs
%\setmainfont[Ligatures=TeX]{Helvetica}   % Police romaine, utilisée dans le corps du document
%\setmainfont[Ligatures=TeX]{Lucida Grande}   % Police romaine, utilisée dans le corps du document



\def\eclaire{\mathbb}

\def\R{\ensuremath{\eclaire R}}
\def\N{\ensuremath{\eclaire N}}
\def\C{\ensuremath{\eclaire C}}
\def\Z{\ensuremath{\eclaire Z}}
\def\Q{\ensuremath{\eclaire Q}}

%\newcommand{\sh}{\mathop{\mathrm{sh}}\nolimits}
\renewcommand{\sinh}{\mathop{\mathrm{sh}}}
\renewcommand{\cosh}{\mathop{\mathrm{ch}}}
\renewcommand{\tanh}{\mathop{\mathrm{th}}}
\renewcommand{\arcsin}{\mathop{\mathrm{Arc\,sin}}}
\renewcommand{\arccos}{\mathop{\mathrm{Arc\,cos}}}
\renewcommand{\arctan}{\mathop{\mathrm{Arc\,tan}}}
\renewcommand{\Im}{\mathop{\mathfrak{I}\mathrm{m}}}
\renewcommand{\Re}{\mathop{\mathfrak{R}\mathrm{m}}}




\DeclareMathOperator{\sh}{sh}
\DeclareMathOperator{\ch}{ch}


\renewcommand{\baselinestretch}{1.5}


% Only one command is required to use Sage within the LaTeX source:
\usepackage{sagetex}

\begin{document} 

\pagecolor{gray!45}

\setlength{\parindent}{0pt}


%\bf \Huge


\title{Les primitives des fonctions de base.}
\date{}


\author{Louis Herzog} 

\maketitle 

\setcounter{tocdepth}{2}
\tableofcontents
%\sommaire


\begin{abstract}
Quelques considérations préliminaires sur le calcul de certaines primitives telles que celles des fonctions 
 $x \mapsto \arccos(x) $,  $x \mapsto \arcsin(x) $ ou bien $x \mapsto \arctan(x) $ une astuce est indispensable. Elle consiste à procéder à une intégration par parties. En effet, n'ayant aucune idée des primitives, il est raisonnable de changer leurs rôles respectifs et de considérer, non plus la fonction initiale, mais sa dérivée.
 
Le premier objectif est donc de vérifier si celles-ci existent et sont calculables.
\end{abstract} 


\section{La fonction  $x \mapsto \arccos(x) $.}


La restriction de la fonction $x \mapsto \cos(x) $ à l'intervalle $[0,\pi]$ est une bijection de $[0,\pi] \rightarrow [-1,1]$ . Il existe donc une fonction réciproque à la fonction $x \mapsto \cos(x) $ que l'on nomme $x \mapsto \arccos(x) $. C'est également une bijection, elle est continue sur l'intervalle fermé  $ [-1,1]$ et est dérivable sur l'intervalle ouvert $]-1,1[$.

\subsection{Calcul de la dérivée de la fonction $x \mapsto \arccos(x) $.}

Pour ce calcul, il faut utiliser le calcul de la dérivée d'une fonction composée. On a $\cos(\arccos(x)=x)$, par conséquent la dérivée de cette expression s'exprime par $ -\sin(\arccos(x) \times (\arccos(x))' = 1$, d'où $(\arccos(x))' = \frac{-1}{\sin(\arccos(x))} $.

La difficulté est maintenant de déterminer $\sin(\arccos(x))$, or on sait que pour tout $X \in \R$, on a $\sin^2(X) + \cos^2(X) = 1$, d'où $\sin(X) = \sqrt{1-\cos^2(X)}$.

En remplaçant $X$ par $\arccos(x)$, 

on a $\sin(\arccos(x)) = \sqrt{1-\cos^2(\arccos(x))} = \sqrt{1- x^2}$.

Finalement, $(\arccos(x))' = \frac{-1}{\sin(\arccos(x))} =  \frac{-1}{\sqrt{1- x^2}} $.

\subsubsection*{Vérification avec Sage}

\begin{sageblock}
    f(x) = acos(x)
    g(x) = diff(f(x),x,1)
\end{sageblock}

La dérivée de $\sage{f(x)} = \sage{g(x)} $.


On peut maintenant entreprendre le calcul de la primitive de la  fonction  $x \mapsto \arccos(x) $.

\subsection{Calcul de la primitive de la fonction  $x \mapsto \arccos(x) $.}




Je pose que $u(x)$  est égal à la fonction $\arccos(x)$ et $v'(x)$ est égal $1$  d'où $u'(x)$  est égal à la fonction $ \frac{-1}{\sqrt{1- x^2}} $ et $v(x)$ est égal $x$.

Alors on a $\int \arccos(x) \, dx = x \times \arccos(x) -\int \frac{-1}{\sqrt{1- x^2}} \times x \, dx $.


\subsubsection*{Calcul de $\int \frac{x}{\sqrt{1- x^2}} \, dx $.}

$\int \frac{x}{\sqrt{1- x^2}} \, dx = \frac{-1}{2} \int \frac{d(1-x^2)}{\sqrt{1- x^2}}= -\sqrt{1- x^2} $.


Finalement $\int \arccos(x) \, dx = x \times \arccos(x) - \sqrt{1- x^2} + Cste $

\subsubsection*{Vérification avec Sage}

\begin{sageblock}
    f(x) = acos(x)
    F(x) = integrate(f(x),x)
\end{sageblock}

La primitive de $\sage{f(x)} = \sage{F(x)} $.


\section{La fonction  $x \mapsto \arcsin(x) $.}

La restriction de la fonction $x \mapsto \sin(x) $ à l'intervalle $[0,\pi]$ est une bijection de $[0,\pi] \rightarrow [-1,1]$ . Il existe donc une fonction réciproque à la fonction $x \mapsto \sin(x) $ que l'on nomme $x \mapsto \arcsin(x) $. C'est également une bijection, elle est continue sur l'intervalle fermé  $ [-1,1]$ et est dérivable sur l'intervalle ouvert $]-1,1[$.

\subsection{Calcul de la dérivée de la fonction $x \mapsto \arcsin(x) $.}



Pour ce calcul, il faut utiliser le calcul de la dérivée d'une fonction composée. On a $\sin(\arcsin(x)=x)$, par conséquent la dérivée de cette expression s'exprime par $ \cos(\arcsin(x) \times (\arcsin(x))' = 1$, d'où $(\arcsin(x))' = \frac{1}{\cos(\arcsin(x))} $.

La difficulté est maintenant de déterminer $\cos(\arcsin(x))$, or on sait que pour tout $X \in \R$, on a $\sin^2(X) + \cos^2(X) = 1$, d'où $\cos(X) = \sqrt{1-\sin^2(X)}$.

En remplaçant $X$ par $\arcsin(x)$, 

on a $\cos(\arcsin(x)) = \sqrt{1-\sin^2(\arcsin(x))} = \sqrt{1- x^2}$.

Finalement, $(\arcsin(x))' = \frac{1}{\cos(\arcsin(x))} =  \frac{1}{\sqrt{1- x^2}} $.

\subsubsection{Vérification avec Sage}

\begin{sageblock}
    f(x) = asin(x)
    g(x) = diff(f(x),x,1)
\end{sageblock}

La dérivée de $\sage{f(x)} = \sage{g(x)} $.


On peut maintenant entreprendre le calcul de la primitive de la  fonction  $x \mapsto \arcsin(x) $.





\subsection{Calcul de la primitive de la fonction  $x \mapsto \arcsin(x) $.}




Je pose que $u(x)$  est égal à la fonction $\arcsin(x)$ et $v'(x)$ est égal $1$  d'où $u'(x)$  est égal à la fonction $ \frac{1}{\sqrt{1- x^2}} $ et $v(x)$ est égal $x$.

Alors on a $\int \arcsin(x) \, dx = x \times \arcsin(x) -\int \frac{1}{\sqrt{1- x^2}} \times x \, dx $.


\subsubsection{Calcul de $\int \frac{x}{\sqrt{1- x^2}} \, dx $.}

$\int \frac{x}{\sqrt{1- x^2}} \, dx = \frac{1}{2} \int \frac{d(1-x^2)}{\sqrt{1- x^2}}= \sqrt{1- x^2} $.


Finalement $\int \arcsin(x) \, dx = x \times \arccos(x) - \sqrt{1- x^2} + Cste $

\subsubsection{Vérification avec Sage}

\begin{sageblock}
    f(x) = asin(x)
    F(x) = integrate(f(x),x)
\end{sageblock}

La primitive de $\sage{f(x)} = \sage{F(x)} $.




\section{La fonction  $x \mapsto \arctan(x) $.}




La restriction de la fonction $x \mapsto \tan(x) $ à l'intervalle $[0,\pi]$ est une bijection de $[0,\pi] \rightarrow [-1,1]$ . Il existe donc une fonction réciproque à la fonction $x \mapsto \tan(x) $ que l'on nomme $x \mapsto \arctan(x) $. C'est également une bijection, elle est continue sur l'intervalle fermé  $ [-1,1]$ et est dérivable sur l'intervalle ouvert $]-1,1[$.



\subsection{Calcul de la dérivée de la fonction $x \mapsto \arctan(x) $.}




Pour ce calcul, il faut utiliser le calcul de la dérivée d'une fonction composée. On a $\tan(\arctan(x)=x)$, par conséquent la dérivée de cette expression s'exprime par $ \tan(\arctan(x)) \times (\arctan(x)' = 1$, d'où $(\arctan(x))' = \frac{1}{1+x^2} $.

La difficulté est maintenant de déterminer $\tan'(\arctan(x)$, or on sait que pour tout $X \in \R$, on a $ \tan'(x) =1+\tan^2(x) $, d'où $\tan'(\arctan(x)) = 1+x^2$.

Finalement, $(\arctan(x))' = \frac{1}{1+x^2}$.

\subsubsection*{Vérification avec Sage}

\begin{sageblock}
    f(x) = atan(x)
    g(x) = diff(f(x),x,1)
\end{sageblock}

La dérivée de $\sage{f(x)} = \sage{g(x)} $.


On peut maintenant entreprendre le calcul de la primitive de la  fonction  $x \mapsto \arctan(x) $.








\subsection{Calcul de la primitive de la fonction  $x \mapsto \arctan(x) $.}



Je pose que $u(x)$  est égal à la fonction $\arctan(x)$ et $v'(x)$ est égal $1$  d'où $u'(x)$  est égal à la fonction $ \frac{1}{1+ x^2} $ et $v(x)$ est égal $x$.

Alors on a $\int \arctan(x) \, dx = x \times \arctan(x) -\int \frac{1}{1+x^2} \times x \, dx $.






\subsubsection{Calcul de $\int \frac{x}{1+ x^2} \, dx $.}

$\int \frac{x}{1+ x^2} \, dx = \frac{1}{2} \int \frac{d(1+x^2)}{1+ x^2} $.


Finalement $\int \arctan(x) \, dx = x \times \arctan(x) -\ln{ \sqrt{1+ x^2}} + Cste $

\subsubsection{Vérification avec Sage}

\begin{sageblock}
    f(x) = atan(x)
    F(x) = integrate(f(x),x)
\end{sageblock}

La primitive de $\sage{f(x)} = \sage{F(x)} $.


\section{La fonction  $x \mapsto \ln(x) $.}

La restriction de la fonction $x \mapsto \tan(x) $ à l'intervalle $[0,\pi]$ est une bijection de $[0,\pi] \rightarrow [-1,1]$ . Il existe donc une fonction réciproque à la fonction $x \mapsto \tan(x) $ que l'on nomme $x \mapsto \arctan(x) $. C'est également une bijection, elle est continue sur l'intervalle fermé  $ [-1,1]$ et est dérivable sur l'intervalle ouvert $]-1,1[$.




\subsection{Calcul de la dérivée de la fonction $x \mapsto \ln(x) $.}


\subsubsection{Première Méthode}


Passons par les limites pour trouver la primitive de $\ln(x)$,

$ \lim_{h \mapsto 0} \frac{\ln(x+h) - \ln(x}{h} = \lim_{h \mapsto 0} \frac{\ln(\frac{x+h}{x})}{h}  = \lim_{h \mapsto 0} \frac{ \ln(1+X)}{x\times X}$, avec $X=\frac{h}{x}$.


On a donc 

$\lim_{h \mapsto 0} \frac{\ln(1+X)}{x\times X} = \frac{1}{x} \times \lim_{h \mapsto 0} \frac{\ln(1+X)}{X} = \frac{1}{x} \times 1 = \frac{1}{x}$




\subsubsection{Seconde Méthode}

Pour ce calcul, il faut utiliser le calcul de la dérivée d'une fonction composée. On a $\exp((\ln(x))=x)$, par conséquent la dérivée de cette expression s'exprime par $ \exp(\ln(x)) \times (\ln(x)' = 1$, d'où $(\ln(x))' = \frac{1}{\exp(\ln(x))}  = \frac{1}{x} $.

\subsubsection{Vérification avec Sage}

\begin{sageblock}
    f(x) = ln(x)
    g(x) = diff(f(x),x,1)
\end{sageblock}

La dérivée de $\sage{f(x)} = \sage{g(x)} $.


On peut maintenant entreprendre le calcul de la primitive de la  fonction  $x \mapsto \ln(x) $.






\subsection{Calcul de la primitive de la fonction  $x \mapsto \ln(x) $.}


Je pose que $u(x)$  est égal à la fonction $\ln(x)$ et $v'(x)$ est égal $1$  d'où $u'(x)$  est égal à la fonction $ \frac{1}{1+ x^2} $ et $v(x)$ est égal $x$.

Alors on a $\int \ln(x) \, dx = x \times \ln(x) -\int \frac{1}{x} \times x \, dx $.


\subsubsection{Calcul de $\int \frac{x}{x} \, dx $.}

$\int \frac{x}{x} \, dx = \int 1 \, dx = x$.


Finalement $\int \ln(x) \, dx = x \times \ln(x) -x + Cste $

\subsubsection{Vérification avec Sage}

\begin{sageblock}
    f(x) = ln(x)
    F(x) = integrate(f(x),x)
\end{sageblock}

La primitive de $\sage{f(x)} = \sage{F(x)} $.


\section{La fonction  $x \mapsto \cos(x)$.}

La restriction de la fonction $x \mapsto \ln(x) $ à l'intervalle $[0,\pi]$ est une bijection de $[0,\pi] \rightarrow [0,+inf]$ . Il existe donc une fonction réciproque à la fonction $x \mapsto \ln(x) $ que l'on nomme $x \mapsto \exp(x) $. C'est également une bijection, elle est continue sur l'intervalle fermé  $ [-1,1]$ et est dérivable sur l'intervalle ouvert $]-inf,+inf[$.



\subsection{Dérivée de la fonction $x \mapsto \cos(x)$.}

\begin{align*}
\lim_{h\to 0} \frac{\cos(x+h)-cos(x)}{h} = &\lim_{h\to 0} \frac{\cos(x)\cos(h)-\sin(x)\sin(h)-cos(x)}{h} \\ =& \lim_{h\to 0} \frac{\cos(x)(\cos(h)-1)}{h}-\frac{\sin(x)\sin(h)}{h}\\  =& -\sin{x}
\end{align*}

\subsubsection{Vérification avec Sage}


\subsection{Calcul de la primitive de la fonction  $x \mapsto \cos(x)$.}


\subsubsection{Vérification avec Sage}


\section{La fonction  $x \mapsto \sin(x)$.}

\subsection{Dérivée de la fonction $x \mapsto \sin(x)$.}


\begin{align*}
\lim_{h\to 0} \frac{\sin(x+h)-sin(x)}{h} =& \lim_{h\to 0} \frac{\cos(x)\sin(h)-\cos(x)\sin(h)-\sin(x)}{h} \\= & \lim_{h\to 0} \frac{\sin(x)(\cos(h)-1)}{h}-\frac{\cos(x)\sin(h)}{h} \\=& \cos{x}
\end{align*}

\subsubsection{Vérification avec Sage}

\subsection{Calcul de la primitive de la fonction  $x \mapsto \sin(x)$.}


\subsubsection{Vérification avec Sage}

\section{La fonction  $x \mapsto \ch(x)$.}

\subsection{Dérivée de la fonction $x \mapsto \ch(x)$.}
\begin{align*}
\ch(x)' =& \left( \frac{\exp(x)+\exp(-x)}{2} \right)' \\ =& \frac{\exp(x)'+\exp(-x)'}{2} \\=& \frac{\exp(x)-\exp(-x)}{2} \\=& \sh(x)
\end{align*}


\subsubsection{Vérification avec Sage}

\subsection{Calcul de la primitive de la fonction  $x \mapsto \ch(x)$.}

\subsubsection{Vérification avec Sage}

\section{La fonction  $x \mapsto \sh(x)$.}



\subsection{Dérivée de la fonction $x \mapsto \sh(x)$.}
\begin{align*}
\sh(x)' =& \left( \frac{\exp(x)-\exp(-x)}{2} \right)' \\ =& \frac{\exp(x)'-\exp(-x)'}{2} \\=& \frac{\exp(x)+\exp(-x)}{2} \\=& \ch(x)
\end{align*}

\subsubsection{Vérification avec Sage}

\subsection{Calcul de la primitive de la fonction  $x \mapsto \sh(x)$.}

\subsubsection{Vérification avec Sage}




%%%%%%%%%%%%%%%%%%%%%%%%%%%%%%%%%%%%%%%


\section{Calcul de quelques primitives.}


\subsection{Calcul d'une primitive de $ x \longmapsto  \frac{dx}{\sqrt{x^2 + 1} } $ }

\subsubsection{Vérification avec Sage}

\subsection{Calcul d'une primitive de $  x \longmapsto  \frac{dx}{\sqrt{x^2+ x + 1} } $ }

\subsubsection{Vérification avec Sage}

\subsection{Calcul d'une primitive de $  x \longmapsto  \frac{dx}{\sqrt{x^2+ \alpha^2} } $ }

\subsubsection{Vérification avec Sage}



\subsubsection{Calcul d'une primitive de $  x \longmapsto  \frac{dx}{\sqrt{x^2+ a^2} } $ par un changement de variable puis par l'emploi d'une variable auxiliaire} 

\subsubsection{Vérification avec Sage}

\subsection{Calcul d'une primitive de $  x \longmapsto  \sqrt{x^2 + 1}  $ \label{sqrt-001} }

\subsubsection{Vérification avec Sage}




\end{document}