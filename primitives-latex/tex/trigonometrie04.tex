% !TEX TS-program = sage 
% !TEX TS-program = xelatex
% !TEX encoding = UTF-8 Unicode
% The following lines are standard LaTeX preamble statements.

%\documentclass[a4paper,12pt]{smfart}

%\documentclass[a4paper,12pt]{smfart}

%\documentclass[a4paper,11pt]{book}
%\documentclass[a4paper,landscape,20pt]{extarticle} %% landscape ou bien portrait (par défault)

%\documentclass[a4paper,17pt]{extarticle} %% on peut utiliser les tailles: 8pt, 9pt,14pt, 17pt et 20pt

%\documentclass[a4paper,11pt]{report}
\documentclass[a4paper,14pt]{extreport} %% on peut utiliser les tailles: 8pt, 9pt,14pt, 17pt et 20pt


\usepackage[french]{babel}
\usepackage[T1]{fontenc}
\usepackage[utf8]{inputenc}

%\usepackage{amsmath,amssymb,amsfonts,url,xspace,smfthm}
\usepackage{amsmath,amsthm,amssymb,amsfonts,url,xspace}
\usepackage{mathtools}
\usepackage[mathscr]{eucal}
%\usepackage{exscale}

\usepackage{diagbox}

\usepackage{fancybox}

\usepackage{xcolor}

%%%%%%%%%%%%%%%%%%%%%%
% - les theoremes
\newtheorem{definition}{D\'efinition}
\newtheorem{theoreme}{Th\'eor\`eme}
\newtheorem{proposition}{Proposition}
\newtheorem{lemme}{Lemme}
\newtheorem{corollaire}{Corollaire}
\newtheorem{propriete}{Propri\'et\'e}
\newtheorem{axiome}{Axiome}
\newcommand{\initialisecompteurs}{
       \setcounter{definition}{0}%
       \setcounter{theoreme}{0}%
       \setcounter{proposition}{0}%
       \setcounter{lemme}{0}%
       \setcounter{corollaire}{0}%
       \setcounter{propriete}{0}%
       \setcounter{axiome}{0}%
       \setcounter{section}{0}%
       \setcounter{subsection}{0}%
       \setcounter{subsubsection}{0}%
}


\usepackage[left=2.cm,right=2.cm,top=2.cm,bottom=2cm,bindingoffset=0cm, headsep=0cm,headheight=0cm]{geometry}

%\usepackage{qrcode}
%\usepackage[draft,forget]{qrcode}
%\usepackage{rmsmacros} 
\usepackage{graphicx}
%\usepackage{hyperref}
%\usepackage[hidelinks]{hyperref}
%\usepackage{tabvar}
\usepackage[tight]{shorttoc}
\newcommand{\sommaire}{\shorttoc{Sommaire}{1}}

\graphicspath{ {./images/} }
\everymath{\displaystyle}

%%%%%%%%%%%%%%%%%%%%%%%%%%%%%%%%

%% Sélection des fontes
%\usepackage{fontspec}  % fontspec et xunicode sont facultatifs
%\setmainfont[Ligatures=TeX]{Helvetica}   % Police romaine, utilisée dans le corps du document
%\setmainfont[Ligatures=TeX]{Lucida Grande}   % Police romaine, utilisée dans le corps du document

\usepackage{ccfonts} 
\usepackage[euler-digits]{eulervm} 

%%%%%%%%%%%%%%%%%%%%%%%%%%%%%%%%


\def\eclaire{\mathbb}

\def\R{\ensuremath{\eclaire R}}
\def\N{\ensuremath{\eclaire N}}
\def\C{\ensuremath{\eclaire C}}
\def\Z{\ensuremath{\eclaire Z}}
\def\Q{\ensuremath{\eclaire Q}}

%\newcommand{\sh}{\mathop{\mathrm{sh}}\nolimits}
\renewcommand{\sinh}{\mathop{\mathrm{sh}}}
\renewcommand{\cosh}{\mathop{\mathrm{ch}}}
\renewcommand{\tanh}{\mathop{\mathrm{th}}}

\renewcommand{\arg}{\mathop{\mathrm{Arg\,}}}

\renewcommand{\arcsin}{\mathop{\mathrm{Arc\,sin}}}
\renewcommand{\arccos}{\mathop{\mathrm{Arc\,cos}}}
\renewcommand{\arctan}{\mathop{\mathrm{Arc\,tan}}}
\renewcommand{\Im}{\mathop{\mathfrak{I}\mathrm{m}}}
\renewcommand{\Re}{\mathop{\mathfrak{R}\mathrm{m}}}


%\DeclareMathOperator{\sh}{sh}
%\DeclareMathOperator{\ch}{ch}


\renewcommand{\baselinestretch}{1.5}


% Only one command is required to use Sage within the LaTeX source:
\usepackage{sagetex}


%%%%%%%%%%%%%%%%%%%%%%%%%%%%%%%%

\newcommand{\limh}[1]{\lim\limits_{h\to 0} \frac{#1(x+h)-#1(x)}{h}}

%%%%%%%%%%%%%%%%%%%%%%%%%%%%%%%%




\begin{document} 

%\pagecolor{gray!45}

\setlength{\parindent}{0pt}


%\bf \Huge

\title{Les primitives et les dérivées.}
\date{}


\author{Louis Herzog} 

\maketitle 

\setcounter{tocdepth}{2}
\tableofcontents
%\sommaire


\begin{abstract}
Quelques considérations préliminaires sur le calcul de certaines primitives telles que celles des fonctions 
 $x \mapsto \arccos(x) $,  $x \mapsto \arcsin(x) $ ou bien $x \mapsto \arctan(x) $ une astuce est indispensable. Elle consiste à procéder à une intégration par parties. En effet, n'ayant aucune idée des primitives, il est raisonnable de changer leurs rôles respectifs et de considérer, non plus la fonction initiale, mais sa dérivée.
 
Le premier objectif est donc de vérifier si celles-ci existent et sont calculables.

Remarque.\\
On passe des formules de trigonométrie aux formules de trigonométries hyperboliques en remplaçant $\cos$ par $\cosh$ et $\sin$ par $i . \sinh$.


\end{abstract} 

\chapter{Fonctions trigonométriques et trigonométriques inverses.}


\section{La fonction  $x \mapsto \cos(x)$.}

La restriction de la fonction $x \mapsto \cos(x) $ à l'intervalle $[0,\pi]$ est une bijection de $[0,\pi] \rightarrow [-1,+1]$ . Il existe donc une fonction réciproque à la fonction $x \mapsto \cos(x) $ que l'on nomme $x \mapsto \arccos(x) $. C'est également une bijection, elle est continue sur l'intervalle fermé  $ [-1,1]$ et est dérivable sur l'intervalle ouvert $ ]-1,1[$.

\subsection{Dérivée de la fonction $x \mapsto \cos(x)$.}

\begin{align*}
\lim_{h\to 0} \frac{\cos(x+h)-cos(x)}{h} = &\lim_{h\to 0} \frac{\cos(x)\cos(h)-\sin(x)\sin(h)-cos(x)}{h} \\ =& \lim_{h\to 0} \frac{\cos(x)(\cos(h)-1)}{h}-\frac{\sin(x)\sin(h)}{h}\\  =&  \cos(x) \times \lim_{h\to 0}\frac{\cos(h)-1}{h}- \sin(x) \times \lim_{h\to 0} \frac{\sin(h)}{h}\\  =& -\sin{x}
\end{align*}

\subsection{Calcul d'une primitive de la fonction  $x \mapsto \cos(x)$.}

\begin{sageblock}
    f(x) = cos(x)
    F(x) = integrate(f(x),x)
\end{sageblock}

Une primitive de la fonction $x \mapsto \sage{f(x)}$  est la fonction $x \mapsto \sage{F(x)} + Cste $ définie à une constante près.


\section{La fonction  $x \mapsto \sin(x)$.}


\subsection{Dérivée de la fonction $x \mapsto \sin(x)$.}

\begin{align*}
\lim_{h\to 0} \frac{\sin(x+h)-sin(x)}{h} =& \lim_{h\to 0} \frac{\cos(x)\sin(h)-\cos(x)\sin(h)-\sin(x)}{h} \\= & \lim_{h\to 0} \frac{\sin(x)(\cos(h)-1)}{h}-\frac{\cos(x)\sin(h)}{h} \\=&\sin(x) \times  \lim_{h\to 0} \frac{\cos(h)-1}{h}-\cos(x) \times \frac{\sin(h)}{h} \\= \cos{x}
\end{align*}

\subsection{Calcul d'une primitive de la fonction  $x \mapsto \sin(x)$.}

\begin{sageblock}
    f(x) = sin(x)
    F(x) = integrate(f(x),x)
\end{sageblock}

Une primitive de $\sage{f(x)}$ est $ \sage{F(x)} + Cste $ définie à une constante près.


\section{La fonction  $x \mapsto \tan(x)$.}

\subsection{Dérivée de la fonction $x \mapsto \tan(x)$.}


\begin{align*}
\tan(x)' = & \left(\frac{\sin(x)}{\cos(x)}\right)' \\ = & \frac{\cos(x) \times \cos(x)+\sin(x)\times \sin(x)}{\cos(x)^2}  \\ = & \frac{1}{\cos(x)^2} = 1 + \tan(x)^2
\end{align*}

\subsubsection{Vérification avec Sage}

\begin{sageblock}
    f(x) = tan(x)
    g(x) = diff(f(x),x)
\end{sageblock}

La dérivée de $\sage{f(x)} = \sage{g(x)} $.


\subsection{Calcul d'une primitive de la fonction  $x \mapsto \tan(x)$.}

On a $\tan(x)=\frac{\sin(x)}{\cos(x)}$, alors $\int \tan(x) \, dx =\int \frac{\sin(x)}{\cos(x)}\, dx$.\\
Je pose $u(x)=\cos(x)$ donc $u'(x)= -\sin(x) \,dx$ et par ce changement de variable on a $\int \tan(x) \, dx = \int \frac{\sin(x)}{\cos(x)}\, dx  = -\int \frac{u'}{u} = -\ln|u|  = \ln\left(\frac{1}{|u|}\right)  = \ln\left(\frac{1}{|\cos(x)|}\right) +Cste$.


\subsubsection*{Vérification avec Sage}

\begin{sageblock}
    f(x) = tan(x)
    F(x) = integrate(f(x),x)
\end{sageblock}

Une primitive de $\sage{f(x)}$ est la fonction définie à une constante près $x \mapsto \sage{F(x)} + Cste $.

La fonction $x\mapsto \sec$ est la fonction paire $x\mapsto \frac{1}{\cos(x)}$ périodique de période $2\pi$ définie sur $\R-\{ \frac{\pi}{2}+k\pi, k\in\Z\}$. On retrouve bien le résultat précédent.


\section{La fonction  $x \mapsto \arccos(x) $.}


La restriction de la fonction $x \mapsto \cos(x) $ à l'intervalle $[0,\pi]$ est une bijection de $[0,\pi] \rightarrow [-1,1]$ . Il existe donc une fonction réciproque à la fonction $x \mapsto \cos(x) $ que l'on nomme $x \mapsto \arccos(x) $. C'est également une bijection, elle est continue sur l'intervalle fermé  $ [-1,1]$ et est dérivable sur l'intervalle ouvert $]-1,1[$.

\subsection{Calcul de la dérivée de la fonction $x \mapsto \arccos(x) $.}

Pour ce calcul, il faut utiliser le calcul de la dérivée d'une fonction composée. On a $\cos(\arccos(x))=x$, par conséquent la dérivée de cette expression s'exprime par $ -\sin(\arccos(x) \times \arccos(x)' = 1$, d'où $\arccos(x)' = \frac{-1}{\sin(\arccos(x))} $.

La difficulté est maintenant de déterminer $\sin(\arccos(x))$, or on sait que pour tout $X \in \R$, on a $\sin^2(X) + \cos^2(X) = 1$, d'où $\sin(X) = \sqrt{1-\cos^2(X)}$.

En remplaçant $X$ par $\arccos(x)$, 
on a $\sin(\arccos(x)) = \sqrt{1-\cos^2(\arccos(x))} = \sqrt{1- x^2}$.

Finalement, $\arccos(x)' = \frac{-1}{\sin(\arccos(x))} = \frac{-1}{\sqrt{1-\cos^2(\arccos(x))}} =  \frac{-1}{\sqrt{1- x^2}} $.

\subsubsection*{Vérification avec Sage}

\begin{sageblock}
    f(x) = arccos(x)
    g(x) = diff(f(x),x)
\end{sageblock}

La dérivée de la fonction $\sage{f(x)}$ est la fonction $x \mapsto \sage{g(x)} $, ce que l'on retrouve sous une écriture légèrement modifiée de Sage.


\begin{center}
\sageplot[width=.45\textwidth]{plot(f(x), x, -1, 1)}
\sageplot[width=.45\textwidth]{plot(cos(x), x, 0, pi)}\\
Les représentations graphiques de $x \mapsto \sage{f(x)} $ et de $x\mapsto \cos(x)$.
\end{center}

On peut maintenant entreprendre le calcul de la primitive de la  fonction  $x \mapsto \arccos(x) $.

\subsection{Calcul de la primitive de la fonction  $x \mapsto \arccos(x) $.}


Je pose que $u(x)$  est égal à la fonction $\arccos(x)$ et $v'(x)$ est égal $1$  d'où $u'(x)$  est égal à la fonction $ \frac{-1}{\sqrt{1- x^2}} $ et $v(x)$ est égal $x$.

Alors on a, par une intégration par parties, $\int \arccos(x) \, dx = x \times \arccos(x) -\int \frac{-1}{\sqrt{1- x^2}} \times x \, dx =  x \arccos(x) + \int \frac{x}{\sqrt{1- x^2}} \, dx $.


\subsubsection*{Calcul de $\int \frac{x}{\sqrt{1- x^2}} \, dx $.}

$\int \frac{x}{\sqrt{1- x^2}} \, dx = \frac{-1}{2} \int \frac{d(1-x^2)}{\sqrt{1- x^2}}= -\sqrt{1- x^2} $.


Finalement $\int \arccos(x) \, dx = x  \arccos(x) - \sqrt{1- x^2} + Cste $ est une primitive de la fonction $x \mapsto \arccos(x) $.

\subsubsection*{Vérification avec Sage}

\begin{sageblock}
    f(x) = arccos(x)
    F(x) = integrate(f(x),x)
\end{sageblock}

Une primitive de $\sage{f(x)} = \sage{F(x)} + Cste$.


\section{La fonction  $x \mapsto \arcsin(x) $.}

La restriction de la fonction $x \mapsto \sin(x) $ à l'intervalle $\left[-\frac{\pi}{2},\frac{\pi}{2}\right]$ est une bijection de $\left[-\frac{\pi}{2},\frac{\pi}{2}\right] \rightarrow [-1,1]$ . Il existe donc une fonction réciproque à la fonction $x \mapsto \sin(x) $ que l'on nomme $x \mapsto \arcsin(x) $. C'est également une bijection, elle est continue sur l'intervalle fermé  $ [-1,1]$ et est dérivable sur l'intervalle ouvert $]-1,1[$.

\subsection{Calcul de la dérivée de la fonction $x \mapsto \arcsin(x) $.}


Pour ce calcul, il faut utiliser le calcul de la dérivée d'une fonction composée. On a $\sin(\arcsin(x))=x$, par conséquent la dérivée de cette expression s'exprime par $ \cos(\arcsin(x)) \times \arcsin(x) ' = 1$, d'où $\arcsin(x) ' = \frac{1}{\cos(\arcsin(x))} $.

La difficulté est maintenant de déterminer $\cos(\arcsin(x))$, or on sait que pour tout $X \in \R$, on a $\sin^2(X) + \cos^2(X) = 1$, d'où $\cos(X) = \sqrt{1-\sin^2(X)}$.

En remplaçant $X$ par $\arcsin(x)$, 

on a $\cos(\arcsin(x)) = \sqrt{1-\sin^2(\arcsin(x))} = \sqrt{1- x^2}$.

Finalement, $\arcsin(x) ' = \frac{1}{\cos(\arcsin(x))}  = \frac{1}{\sqrt{1-\sin^2(\arcsin(x))}} =  \frac{1}{\sqrt{1- x^2}} $.

\subsubsection{Vérification avec Sage}

\begin{sageblock}
    f(x) = arcsin(x)
    g(x) = diff(f(x),x)
\end{sageblock}

La dérivée de $\sage{f(x)} = \sage{g(x)} $.


\begin{center}
\sageplot[width=.45\textwidth]{plot(f(x), x, -1, 1)}
\sageplot[width=.45\textwidth]{plot(sin(x), x, -pi/2, pi/2)}\\
Les représentations graphiques de $x \mapsto \sage{f(x)} $ et de $x\mapsto \sin(x)$.
\end{center}


On peut maintenant entreprendre le calcul de la primitive de la  fonction  $x \mapsto \arcsin(x) $.

\subsection{Calcul de la primitive de la fonction  $x \mapsto \arcsin(x) $.}


Je pose que $u(x)$  est égal à la fonction $\arcsin(x)$ et $v'(x)$ est égal $1$  d'où $u'(x)$  est égal à la fonction $\arcsin(x) ' = \frac{1}{\sqrt{1- x^2}} $ et $v(x)$ est égal $x$.

Alors on a $\int \arcsin(x) \, dx = x \times \arcsin(x) -\int \frac{1}{\sqrt{1- x^2}} \times x \, dx $.


\subsubsection{Calcul de $\int \frac{x}{\sqrt{1- x^2}} \, dx $.}

$\int \frac{x}{\sqrt{1- x^2}} \, dx = \frac{1}{2} \int \frac{d(1-x^2)}{\sqrt{1- x^2}}= \sqrt{1- x^2} $.


Finalement, une primitive de la fonction $x \mapsto \arcsin(x) $ est une fonction  $ x \mapsto x \arcsin(x) - \sqrt{1- x^2} + Cste $.

\subsubsection{Vérification avec Sage}

\begin{sageblock}
    f(x) = arcsin(x)
    F(x) = integrate(f(x),x)
\end{sageblock}

Une primitive de la fonction $\sage{f(x)} = \sage{F(x)} + Cste$.


\section{La fonction  $x \mapsto \arctan(x) $.}


La restriction de la fonction $x \mapsto \tan(x) $ à l'intervalle $\left[-\frac{\pi}{2},\frac{\pi}{2}\right]$ est une bijection de $\left[-\frac{\pi}{2},\frac{\pi}{2}\right] \rightarrow \R $. Il existe donc une fonction réciproque à la fonction $x \mapsto \tan(x) $ que l'on nomme $x \mapsto \arctan(x) $. C'est également une bijection, elle est continue sur l'intervalle fermé  $ [-1,1]$ et est dérivable sur l'intervalle ouvert $]-1,1[$.


\subsection{Calcul de la dérivée de la fonction $x \mapsto \arctan(x) $.}


Pour ce calcul, il faut utiliser le calcul de la dérivée d'une fonction composée. On a $\tan(\arctan(x))=x$, par conséquent la dérivée de cette expression s'exprime par $ \tan'(\arctan(x)) \times \arctan(x)' = 1$, d'où $\arctan(x)' = \frac{1}{\tan'(\arctan(x))} $.

La difficulté est maintenant de déterminer $\tan'(\arctan(x)$, or on sait que pour tout $X \in \R$, on a $ \tan'(x) =1+\tan^2(x) $, d'où $\tan'(\arctan(x)) = 1+x^2$.

Finalement, $\arctan(x)' = \frac{1}{1+x^2}$.

\subsubsection*{Vérification avec Sage}

\begin{sageblock}
    f(x) = arctan(x)
    g(x) = diff(f(x),x)
\end{sageblock}

La dérivée de $\sage{f(x)} = \sage{g(x)} $.


\begin{center}
\sageplot[width=.45\textwidth]{plot(f(x), x, -10, 10)} 
\sageplot[width=.45\textwidth]{plot(tan(x), x, -1.4, 1.4)}\\
Les représentations graphiques de $x \mapsto \sage{f(x)} $ et de $x\mapsto \tan(x)$.
\end{center}


On peut maintenant entreprendre le calcul de la primitive de la  fonction  $x \mapsto \arctan(x) $.


\subsection{Calcul de la primitive de la fonction  $x \mapsto \arctan(x) $.}


Je pose que $u(x)$  est égal à la fonction $\arctan(x)$ et $v'(x)$ est égal $1$  d'où $u'(x)$  est égal à la fonction $ \frac{1}{1+ x^2} $ et $v(x)$ est égal $x$.

Alors on a $\int \arctan(x) \, dx = x \times \arctan(x) -\int \frac{1}{1+x^2} \times x \, dx $.

\subsubsection{Calcul de $\int \frac{x}{1+ x^2} \, dx $.}

$\int \frac{x}{1+ x^2} \, dx = \frac{1}{2} \int \frac{d(1+x^2)}{1+ x^2} $.

D'où $\int \arctan(x) \, dx = x \arctan(x) - \frac{1}{2} \ln \left| 1+ x^2 \right| + Cste $. 
Finalement, une primitive de la fonction $x \mapsto \arctan(x) $ est une fonction $x \mapsto x \arctan(x) -\ln\left( \sqrt{1+ x^2}\right) + Cste $ ou encore $x \mapsto x \arctan(x) +\ln\left( \frac{1}{\sqrt{1+ x^2}}\right) + Cste $.


\subsubsection{Vérification avec Sage}

\begin{sageblock}
    f(x) = arctan(x)
    F(x) = integrate(f(x),x)
\end{sageblock}

Une primitive de $\sage{f(x)} = \sage{F(x)} + Cste$.

%%%%%%%%%%%%%%%%%%%%%%%%%%%%%%%%%%%%%%

\chapter{Fonctions hyperboliques et hyperboliques inverses.}


On passe des formules de trigonométrie aux formules de trigonométries hyperboliques en remplaçant $\cos$ par $\cosh$ et $\sin$ par $i . \sinh$. Par exemple pour $\cos^2+\sin^2=1$
nous obtenons $(\cosh)^2 + (i . \sinh)^2= (\cosh)^2 - (\sinh)^2 = 1$ et pour $\cos(a+b)=\cos(a) \cos(b) - \sin(a) \sin(b) $, nous obtenons $\cosh(a+b)=\cosh(a) \cosh(b) - i .\sinh(a) i . \sinh(b) $ c'est-à-dire $\cosh(a+b)=\cosh(a) \cosh(b) - (i)^2 \sinh(a) \sinh(b) $. \\ Finalement on a $\cosh(a+b) = \cosh(a) \cosh(b) + \sinh(a) \sinh(b) $. On change de signe!



\section{La fonction  $x \mapsto \cosh(x)$.}

\subsection{Dérivée de la fonction $x \mapsto \cosh(x)$.}
\begin{align*}
\cosh(x)' =& \left( \frac{\exp(x)+\exp(-x)}{2} \right)' \\ =& \frac{\exp(x)'+\exp(-x)'}{2} \\=& \frac{\exp(x)-\exp(-x)}{2} \\=& \sinh(x)
\end{align*}




\subsection{Calcul d'une primitive de la fonction  $x \mapsto \cosh(x)$.}

$\int \cosh(x) dx = \int \frac{\exp(x)+ \exp(-x)}{2} dx = \frac{1}{2} \times \int \exp(x) dx + \int \exp(x) dx = \frac{1}{2} \times \exp(x) - \exp(-x) = \sinh$
\\


\section{La fonction  $x \mapsto \sinh(x)$.}


\subsection{Dérivée de la fonction $x \mapsto \sinh(x)$.}

\begin{align*}
\sinh(x)' =& \left( \frac{\exp(x)-\exp(-x)}{2} \right)' \\ =& \frac{\exp(x)'-\exp(-x)'}{2} \\=& \frac{\exp(x)+\exp(-x)}{2} \\=& \cosh(x)
\end{align*}





\subsection{Calcul d'une primitive de la fonction  $x \mapsto \sinh(x)$.}


$\int \sinh(x) dx = \int \frac{\exp(x)- \exp(-x)}{2} dx = \frac{1}{2} \times \int \exp(x) dx - \int \exp(x) dx = \frac{1}{2} \times \exp(x) + \exp(-x) = \cosh$


\subsubsection{Vérification avec Sage}

\begin{sageblock}
    f(x) = sinh(x)
    F(x) = integrate(f(x),x)
\end{sageblock}


Une primitive de $\sage{f(x)} = \sage{F(x)} $.

Le graphe de $\sage{f(x)} $.


\begin{center}
\sageplot[width=.5\textwidth]{plot(f(x), x, -5, 5)} \\
%\sageplot{plot(g(x), x, -1, 1)}
\end{center}




\section{La fonction  $x \mapsto \tanh(x)$.}

\subsection{Dérivée de la fonction $x \mapsto \tanh(x)$.}

\begin{align*}
(\tanh(x))' = & \frac{\cosh(x)'}{\sinh(x)'} \\  = & \frac{\cosh(x)' \times \sinh(x) - \sinh(x)' \times \cosh(x)}{\cosh(x)^2} \\ = &  \frac{sinh(x)^2 -\cosh(x)^2}{\cosh(x)^2} \\ = & \frac{1}{\cosh(x)^2}
\end{align*}


\subsubsection{Vérification avec Sage}


\subsection{Calcul d'une primitive de la fonction  $x \mapsto \tanh(x)$.}

\begin{align*}
\int \tanh(x) = &\int \frac{\cosh(x)}{\sinh(x)} \\  = & \int \frac{1}{u(x)} \times du(x), u(x) =  \cosh(x), du(x) = \sinh(x) \\ = & \ln(u(x)) \\ = & \ln(\cosh(x))
\end{align*}


\subsubsection{Vérification avec Sage}


\begin{sageblock}
    f(x) = tanh(x)
    F(x) = integrate(f(x),x)
\end{sageblock}

Une primitive de $\sage{f(x)} = \sage{F(x)} $.

Le graphe de $\sage{f(x)} $.


\begin{center}
\sageplot[width=.5\textwidth]{plot(f(x), x, -5, 5)} \\
%\sageplot{plot(g(x), x, -1, 1)}
\end{center}

















\section{La fonction  $x \mapsto \arg \sinh(x)$.}

\subsection{Dérivée de la fonction $x \mapsto \arg \sinh(x$.}

\begin{align*}

\end{align*}


\subsubsection{Vérification avec Sage}


\subsection{Calcul d'une primitive de la fonction  $x \mapsto \arg \sinh(x$.}

\begin{align*}
\end{align*}


\subsubsection{Vérification avec Sage}



Une primitive de $\sage{f(x)} = \sage{F(x)} $.

Le graphe de $\sage{f(x)} $.
























%%%%%%%%%%%%%%%%%%%%%%%%%%%%%%%%%%%%%%%



\chapter{Calcul de quelques primitives.}

\section{La fonction  $x \mapsto \ln(x) $.}

\begin{align*}
\int \ln(x) = & \int \ln(x) \times 1 \\ = & x \times \ln(x) - \int \ln(x)' \times x dx \\ = & x \times \ln(x) - \int 1 dx \\ = &  x \times \ln(x) - x
\end{align*}


\section{La fonction  $x \mapsto \exp(x) $.}


\begin{align*}
\int \exp(x) = \exp(x)
\end{align*}






\section{Calcul d'une primitive de $ x \longmapsto  \frac{dx}{\sqrt{x^2 + 1} } $ }


$\int \frac{dx}{\sqrt{x^2 + 1}}$. Je pose $y+x = \sqrt{x^2+1} $


\section{Calcul d'une primitive de $  x \longmapsto  \frac{dx}{\sqrt{x^2+ x + 1} } $ }




\section{Calcul d'une primitive de $  x \longmapsto  \frac{dx}{\sqrt{x^2+ \alpha^2} } $ }

\subsection{Vérification avec Sage}



\subsubsection{Calcul d'une primitive de $  x \longmapsto  \frac{dx}{\sqrt{x^2+ a^2} } $ par un changement de variable puis par l'emploi d'une variable auxiliaire} 

\subsubsection{Vérification avec Sage}

\section{Calcul d'une primitive de $  x \longmapsto  \sqrt{x^2 + 1}  $ \label{sqrt-001} }

\subsubsection{Vérification avec Sage}



\subsection{Calcul de la dérivée de la fonction $x \mapsto \ln(x) $.}


\subsubsection{Première Méthode}


Passons par les limites pour trouver Une primitive de $\ln(x)$,

$ \lim_{h \mapsto 0} \frac{\ln(x+h) - \ln(x}{h} = \lim_{h \mapsto 0} \frac{\ln(\frac{x+h}{x})}{h}  = \lim_{h \mapsto 0} \frac{ \ln(1+X)}{x\times X}$, avec $X=\frac{h}{x}$.


On a donc 

$\lim_{h \mapsto 0} \frac{\ln(1+X)}{x\times X} = \frac{1}{x} \times \lim_{h \mapsto 0} \frac{\ln(1+X)}{X} = \frac{1}{x} \times 1 = \frac{1}{x}$




\subsubsection{Seconde Méthode}

Pour ce calcul, il faut utiliser le calcul de la dérivée d'une fonction composée. On a $\exp((\ln(x))=x)$, par conséquent la dérivée de cette expression s'exprime par $ \exp(\ln(x)) \times (\ln(x)' = 1$, d'où $(\ln(x))' = \frac{1}{\exp(\ln(x))}  = \frac{1}{x} $.

\subsubsection{Vérification avec Sage}

\begin{sageblock}
    f(x) = ln(x)
    g(x) = diff(f(x),x,1)
\end{sageblock}

La dérivée de $\sage{f(x)} = \sage{g(x)} $.

Le graphe de $\sage{f(x)} $.


\begin{center}
\sageplot[width=.5\textwidth]{plot(f(x), x, 0, 10)} \\
%\sageplot{plot(g(x), x, -1, 1)}
\end{center}







On peut maintenant entreprendre le calcul d'une primitive de la  fonction  $x \mapsto \ln(x) $.






\subsection{Calcul d"une primitive de la fonction  $x \mapsto \ln(x) $.}


Je pose que $u(x)$  est égal à la fonction $\ln(x)$ et $v'(x)$ est égal $1$  d'où $u'(x)$  est égal à la fonction $ \frac{1}{1+ x^2} $ et $v(x)$ est égal $x$.

Alors on a $\int \ln(x) \, dx = x \times \ln(x) -\int \frac{1}{x} \times x \, dx $.


\subsubsection{Calcul de $\int \frac{x}{x} \, dx $.}

$\int \frac{x}{x} \, dx = \int 1 \, dx = x$.


Finalement $\int \ln(x) \, dx = x \times \ln(x) -x + Cste $

\subsubsection{Vérification avec Sage}

\begin{sageblock}
    f(x) = log(x)
    F(x) = integrate(f(x),x)
\end{sageblock}

Une primitive de $\sage{f(x)} = \sage{F(x)} $.





\end{document}