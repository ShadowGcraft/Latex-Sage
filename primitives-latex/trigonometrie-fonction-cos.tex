% !TeX document-id = {3e194d8f-a749-4982-ad60-c4185736f4de}
% !TEX TS-program = sage 
% !TEX TS-program = xelatex
% !TEX encoding = UTF-8 Unicode
% The following lines are standard LaTeX preamble statements.


%\documentclass[a4paper,12pt]{smfart}

%\documentclass[a4paper,11pt]{book}
%\documentclass[a4paper,landscape,20pt]{extarticle} %% landscape ou bien portrait (par défault)

%\documentclass[a4paper,17pt]{extarticle} %% on peut utiliser les tailles: 8pt, 9pt,14pt, 17pt et 20pt

\documentclass[a4paper,12pt]{report}
%\documentclass[a4paper,landscape,17pt]{extreport} %% on peut utiliser les tailles: 8pt, 9pt,14pt, 17pt et 20pt


\usepackage[french]{babel}
\usepackage[T1]{fontenc}
\usepackage[utf8]{inputenc}

%\usepackage{amsmath,amssymb,amsfonts,url,xspace,smfthm}
\usepackage{amsmath,amsthm,amssymb,amsfonts,url,xspace}
\usepackage{mathtools}
\usepackage[mathscr]{eucal}
%\usepackage{exscale}

\usepackage{diagbox}

\usepackage{fancybox}

\usepackage{xcolor}

%%%%%%%%%%%%%%%%%%%%%%
% - les theoremes
\newtheorem{definition}{D\'efinition}
\newtheorem{theoreme}{Th\'eor\`eme}
\newtheorem{proposition}{Proposition}
\newtheorem{lemme}{Lemme}
\newtheorem{corollaire}{Corollaire}
\newtheorem{propriete}{Propri\'et\'e}
\newtheorem{axiome}{Axiome}
\newcommand{\initialisecompteurs}{
       \setcounter{definition}{0}%
       \setcounter{theoreme}{0}%
       \setcounter{proposition}{0}%
       \setcounter{lemme}{0}%
       \setcounter{corollaire}{0}%
       \setcounter{propriete}{0}%
       \setcounter{axiome}{0}%
       \setcounter{section}{0}%
       \setcounter{subsection}{0}%
       \setcounter{subsubsection}{0}%
}

\usepackage[left=1.5cm, right=1.5cm, top=1.5cm, bottom=2.cm, bindingoffset=0.cm, headsep=0.cm, headheight=0.cm]{geometry}

%\usepackage{qrcode}
%\usepackage[draft,forget]{qrcode}
%\usepackage{rmsmacros} 
\usepackage{graphicx}
%\usepackage{hyperref}
%\usepackage[hidelinks]{hyperref}
%\usepackage{tabvar}
\usepackage[tight]{shorttoc}
\newcommand{\sommaire}{\shorttoc{Sommaire}{1}}

\graphicspath{ {./images/} }
\everymath{\displaystyle}

%%%%%%%%%%%%%%%%%%%%%%%%%%%%%%%%

%\usepackage[boldsans]{ccfonts}

%\usepackage{concmath}

\usepackage{cmbright}

%\usepackage[euler-digits]{eulervm} 


%\usepackage{beton,euler}

%%%%%%%%%%%%%%%%%%%%%%%%%%%%%%%%


\def\eclaire{\mathbb}

\def\R{\ensuremath{\eclaire R}}
\def\N{\ensuremath{\eclaire N}}
\def\C{\ensuremath{\eclaire C}}
\def\Z{\ensuremath{\eclaire Z}}
\def\Q{\ensuremath{\eclaire Q}}

\def\Id{\ensuremath{\mathrm{Id}}}

%\newcommand{\sh}{\mathop{\mathrm{sh}}\nolimits}
\renewcommand{\sinh}{\mathop{\mathrm{sh}}}
\renewcommand{\cosh}{\mathop{\mathrm{ch}}}
\renewcommand{\tanh}{\mathop{\mathrm{th}}}

\renewcommand{\arg}{\mathop{\mathrm{Arg}}}

\renewcommand{\arcsin}{\mathop{\mathrm{Arc\mspace{2mu}sin}}}
\renewcommand{\arccos}{\mathop{\mathrm{Arc\mspace{2mu}cos}}}
\renewcommand{\arctan}{\mathop{\mathrm{Arc\mspace{2mu}tan}}}
\renewcommand{\Im}{\mathop{\mathfrak{I}\mathrm{m}}}
\renewcommand{\Re}{\mathop{\mathfrak{R}\mathrm{m}}}


%\DeclareMathOperator{\sh}{sh}
%\DeclareMathOperator{\ch}{ch}


\renewcommand{\baselinestretch}{1.5}


% Only one command is required to use Sage within the LaTeX source:
\usepackage{sagetex}


%%%%%%%%%%%%%%%%%%%%%%%%%%%%%%%%

\newcommand{\limh}[1]{\lim\limits_{h\to 0} \frac{#1(x+h)-#1(x)}{h}}

%%%%%%%%%%%%%%%%%%%%%%%%%%%%%%%%




\begin{document} 

%\pagecolor{gray!45}

\setlength{\parindent}{0pt}


%\bf \Huge

\title{Les primitives et les dérivées.}
%\date{Ce document est destiné à un affichage sur un écran et non pas à une impression sur papier.}


\author{Louis Herzog} 

\maketitle 

\setcounter{tocdepth}{2}
\tableofcontents
%\sommaire




\section{La fonction  $x \mapsto \cos(x)$.}
Définissons nos fonctions dans {\texttt{Sage}}. Soit
\begin{sageblock}
    f(x) = cos(x)
    g(x) = diff(f(x),x)
    F(x) = integrate(f(x),x)
\end{sageblock}
\begin{center}
\sageplot[width=.3\textwidth]{plot(cos(x), x, -pi, pi)} \\
La représentation graphique de $x\mapsto \cos(x)$ sur l'intervalle $[-\pi, \pi]$.
\end{center}
La fonction est paire et périodique de période $2 \pi$.

\subsection{Dérivée de la fonction $x \mapsto \cos(x)$.}
\begin{align*}
\lim_{h\to 0} \frac{\cos(x+h)-cos(x)}{h}  
& = \lim_{h\to 0} \frac{\cos(x)\cos(h)-\sin(x)\sin(h)-cos(x)}{h} \\ 
& = \lim_{h\to 0} \left( \frac{\cos(x)(\cos(h)-1)}{h}-\frac{\sin(x)\sin(h)}{h} \right) \\  
& =  \cos(x) \times \lim_{h\to 0}\frac{\cos(h)-1}{h}- \sin(x) \times \lim_{h\to 0} \frac{\sin(h)}{h}\\  
& = -\sin{x}
\end{align*}

\subsection{Calcul d'une primitive de la fonction  $x \mapsto \cos(x)$.}
Dans la section suivante, on calcule la dérivée de la fonction $x \mapsto \sin(x)$ qui vaut $x \mapsto \cos(x)$, par conséquent une primitive de $x \mapsto \cos(x)$ est égale, à une constante près, à $  \sin(x) + C^{ste} $.
\paragraph{On vérifie ce résultat avec Sage.}
Une primitive de la fonction $x \mapsto \sage{f(x)}$  est la fonction $x \mapsto \sage{F(x)} + C^{ste} $ définie à une constante près.



\section{La fonction  $x \mapsto \arccos(x) $.}
La restriction de la fonction $x \mapsto \cos(x) $ à l'intervalle $[0,\pi]$ est une bijection de $[0,\pi] \rightarrow [-1,1]$ . Il existe donc une fonction réciproque à la fonction $x \mapsto \cos(x) $ que l'on nomme $x \mapsto \arccos(x) $. C'est également une bijection, elle est continue sur l'intervalle fermé  $ [-1,1]$ et est dérivable sur l'intervalle ouvert $]-1,1[$.

\subsection{Calcul de la dérivée de la fonction $x \mapsto \arccos(x) $.}
Définissons nos fonctions dans {\texttt{Sage}}
\begin{sageblock}
    f(x) = arccos(x)
    g(x) = diff(f(x),x)
    F(x) = integrate(f(x),x)
\end{sageblock}

Pour ce faire, il faut utiliser le calcul de la dérivée d'une fonction composée. On a $\cos(\arccos(x))=x$, par conséquent la dérivée de cette expression s'exprime par $ -\sin(\arccos(x) \times \arccos\nolimits'(x) = 1$, d'où $\arccos(x)' = \frac{-1}{\sin(\arccos(x))} $.

La difficulté est maintenant de déterminer $\sin(\arccos(x))$, or on sait que pour tout $X \in \R$, on a $\sin^2(X) + \cos^2(X) = 1$, d'où $\sin(X) = \sqrt{1-\cos^2(X)}$.

En remplaçant $X$ par $\arccos(x)$, 
on a $\sin(\arccos(x)) = \sqrt{1-\cos^2(\arccos(x))} = \sqrt{1- x^2}$.

Finalement, $\arccos\nolimits'(x) = \frac{-1}{\sin(\arccos(x))} = \frac{-1}{\sqrt{1-\cos^2(\arccos(x))}} =  \frac{-1}{\sqrt{1- x^2}} $.
\paragraph{On vérifie ce résultat avec Sage.}
La dérivée de la fonction $\sage{f(x)}$ est la fonction $x \mapsto \sage{g(x)} $, ce que l'on retrouve sous une écriture légèrement modifiée de Sage.


\begin{center}
\sageplot[width=.3\textwidth]{plot(f(x), x, -1, 1)}
\sageplot[width=.3\textwidth]{plot(cos(x), x, 0, pi)}\\
Les représentations graphiques de $x \mapsto \sage{f(x)} $ et de $x\mapsto \cos(x)$.
\end{center}

On peut maintenant entreprendre le calcul de la primitive de la  fonction  $x \mapsto \arccos(x) $.

\subsection{Calcul de la primitive de la fonction  $x \mapsto \arccos(x) $.}
Je pose que $u(x)$  est égal à la fonction $\arccos(x)$ et $v'(x)$ est égal $dx$  d'où $u'(x)$  est égal à la fonction $ \frac{-1}{\sqrt{1- x^2}} $ et $v(x)$ est égal $x$.
Alors on a, par une intégration par parties, $\int \arccos(x) \, dx = x \times \arccos(x) -\int \frac{-1}{\sqrt{1- x^2}} \times x \, dx =  x \arccos(x) + \int \frac{x}{\sqrt{1- x^2}} \, dx $.
\paragraph{Calcul de $\int \frac{x}{\sqrt{1- x^2}} \, dx $.}
$\int \frac{x}{\sqrt{1- x^2}} \, dx = \frac{-1}{2} \int \frac{d(1-x^2)}{\sqrt{1- x^2}}= -\sqrt{1- x^2} $.

Finalement, $\int \arccos(x) \, dx = x  \arccos(x) - \sqrt{1- x^2} + C^{ste} $ est une primitive de la fonction $x \mapsto \arccos(x) $.
\paragraph{On vérifie ce résultat avec Sage.}
Une primitive de $\sage{f(x)} = \sage{F(x)} + C^{ste}$.





%%%%%%%%%%%%%%%%%%%%%%%%%%%%%%%%%%
\section{La fonction  $x \mapsto \cosh(x)$.}

Définissons nos fonctions dans {\texttt{Sage}}
\begin{sageblock}
    f(x) = cosh(x)
    g(x) = diff(f(x),x)
    F(x) = integrate(f(x),x)
\end{sageblock}



\begin{center}
\sageplot[width=.3\textwidth]{plot(f(x), x, -4, 4)}
\sageplot[width=.3\textwidth]{plot(g(x), x, -4, 4)} \\
La représentation graphique de $x \mapsto \sage{f(x)} $ et de sa dérivée. 
\end{center}

\subsection{Dérivée de la fonction $x \mapsto \cosh(x)$.}
\begin{align*}
\cosh(x)' & = \left( \frac{\exp(x)+\exp(-x)}{2} \right)' \\ 
& = \frac{\exp(x)'+\exp(-x)'}{2} \\
& = \frac{\exp(x)-\exp(-x)}{2} \\
& = \sinh(x)
\end{align*}

\subsection{Calcul d'une primitive de la fonction  $x \mapsto \cosh(x)$.}
$\int \cosh(x) dx = \int \frac{\exp(x)+ \exp(-x)}{2} dx = \frac{1}{2} \times \int \exp(x) dx + \frac{1}{2} \times \int \exp(-x) dx = \frac{ \exp(x) - \exp(-x) }{2} = \sinh + C^{ste}$
\paragraph{On vérifie ce résultat avec Sage.}
Une primitive de $\sage{f(x)} = \sage{F(x)} + C^{ste}$.








%%%%%%%%%%%%%%%%%%%%%%%%%%%%%%%%%%
\section{La fonction  $x \mapsto \arg\mspace{-1mu}\cosh(x)$.}
Définissons nos fonctions dans {\texttt{Sage}}
\begin{sageblock}
    f(x) = arccosh(x)
    g(x) = diff(f(x),x)
    F(x) = integrate(f(x),x)
\end{sageblock}

Le cosinus hyperbolique, noté $\cosh$ est défini sur $\R$ selon l'expression $\frac{\exp(x)+\exp(-x)}{2}$, son domaine de valeurs est $[1, +\infty [$ 
c'est une fonction paire c'est-à-dire $\cosh(-x)=\cosh(x)$.


La fonction $x \mapsto \cosh(x)$ est inversible sur le domaine de définition restreint à $\R^+$, car elle y est bijective, son inverse est notée \og $ \arg\mspace{-1mu}\cosh $\fg et définit la fonction \og\emph{argument cosinus hyperbolique}\fg telle que $x \mapsto \arg\mspace{-1mu}\cosh(x)$.
\begin{center}
\sageplot[width=.3\textwidth]{plot(f(x), x, 1, 20)} \\
La représentation graphique de $x \mapsto \arg\mspace{-1mu}\cosh(x)$.
\end{center}
On observe que la fonction est croissante, continue sur $\left[1\,,\,+\infty \right[ $ et dérivable sur l'intervalle ouvert $\left]1\,,\,+\infty \right[ $.
                                                                                               

\subsection{Dérivée de la fonction $x \mapsto \arg\mspace{-1mu}\cosh(x)$.\label{argcosh}}
On a la fonction composée $\Id = \cosh \circ \arg\mspace{-1mu}\cosh$ telle que $x\mapsto\cosh\left(\arg\mspace{-1mu}\cosh(x)\right)=x$ dont la dérivée s'écrit alors $1= \arg\mspace{-1mu}\cosh\nolimits' \times \cosh\nolimits'\circ\arg\mspace{-1mu}\cosh$.
\begin{align*}
x & =\cosh\left(\arg\mspace{-1mu}\cosh(x)\right)(x) \quad \textrm{en dérivant, on a}\\
1 & = \arg\mspace{-1mu}\cosh\nolimits'(x) \times \cosh\nolimits'\circ\arg\mspace{-1mu}\cosh(x) \quad \textrm{d'où}\\
\arg\mspace{-1mu}\cosh\nolimits'(x) & =\frac{1}{ \cosh\nolimits'\circ\arg\mspace{-1mu}\cosh(x)}=\frac{1}{ \sinh\left(\arg\mspace{-1mu}\sinh(x)\right)}\quad \textrm{or, on sait que}\\
1 & = \cosh\nolimits^2\left(\arg\mspace{-1mu}\cosh(x)\right) - \sinh\nolimits^2\left(\arg\mspace{-1mu}\cosh(x)\right) \quad \textrm{alors}\\
\sinh\left(\arg\mspace{-1mu}\cosh(x)\right) & = \sqrt{\cosh\nolimits^2\left(\arg\mspace{-1mu}\cosh(x)\right)-1}=\sqrt{x^2-1}\quad \textrm{finalement}\\
\arg\mspace{-1mu}\cosh\nolimits'(x) & = \frac{1}{ \sqrt{x^2-1}}\quad \textrm{on vérifie ce calcul avec Sage.}
\end{align*}


\paragraph{On vérifie ce résultat avec Sage.}
La dérivée de $\sage{f(x)} = \sage{g(x)} $.


\subsection{Calcul d'une primitive de la fonction  $x \mapsto \arg\mspace{-1mu}\cosh(x)$.}
Pour calculer $\int \arg\mspace{-1mu}\cosh(x) \, dx$, on procède par une intégration par parties en posant $u(x) = \arg\mspace{-1mu}\cosh(x)$ et $v'(x) = dx$, d'où $u'(x) = \frac{1}{ \sqrt{x^2-1}}$ et $ v(x) = x $. \\
On a donc
\begin{align*}
\int \arg\mspace{-1mu}\cosh(x) \, dx & = x \arg\mspace{-1mu}\cosh(x) - \int \frac{x}{ \sqrt{x^2-1}} \, dx \quad \textrm{or}\\
\int \frac{x}{ \sqrt{x^2-1}} \, dx & = \int \left(\sqrt{x^2-1}\right)' \, dx = \sqrt{x^2-1}  \quad \textrm{d'où}\\
\int \arg\mspace{-1mu}\cosh(x) \, dx & = x \arg\mspace{-1mu}\cosh(x) - \sqrt{x^2-1} + C^{ste} 
\end{align*}


\paragraph{On vérifie ce résultat avec Sage.}
Une primitive de $\sage{f(x)} = \sage{F(x)} + C^{ste} $.

\begin{center}
\sageplot[width=.3\textwidth]{plot(f(x), x, 1, 20)} 
\sageplot[width=.3\textwidth]{plot(g(x), x, 1, 20)} 
\sageplot[width=.3\textwidth]{plot(F(x), x, 1, 20)}\\
Les représentations graphiques respectivement de $x \mapsto \arg\mspace{-1mu}\cosh(x)$, de sa dérivée et de sa primitive.
\end{center}









%%%%%%%%%%%%%%%%%%%%%%%%%%%%%%%
\section{Calcul d'une primitive de $ x \longmapsto  \frac{dx}{\sqrt{x^2 - 1} } $. }


Je pose $y-x = \sqrt{x^2 - 1} $ avec $y - x = \sqrt{x^2 - 1} \geqslant 0 $, donc $y \geqslant  x$.

Quel est le signe de $y$ c'est-à-dire le signe de $x + \sqrt{x^2 - 1} $?

\begin{center}
\sageplot[width=.4\textwidth]{plot(x+sqrt(x^2 - 1), x, -5, 5)} 
\sageplot[width=.4\textwidth]{plot(diff(x+sqrt(x^2 - 1),x), x, -5, 5)} \\
La représentation graphique de $x + \sqrt{x^2 - 1}$ et de sa dérivée.
\end{center}



En élevant au carré, on a 
\begin{align*}
(y-x)^2 & = x^2 - 1 \\
y^2 - 2yx + x^2 & = x^2 - 1 \\
y^2 - 2yx  & = -1 \quad \textrm{ puis en différentiant chaque variable} \\
2ydy - 2dy \times x - 2ydx & = 0 \\
(y-x)dy & = ydx \\
\frac{dy}{y} & = \frac{dx}{(y-x)} = \frac{dx}{\sqrt{x^2 - 1} } \\
\int \frac{dy}{y} = \ln|y| + C^{ste} & = \int \frac{dx}{\sqrt{x^2 - 1} } dx \quad \textrm{or} \; y = x + \sqrt{x^2 - 1}  \geqslant 0 \\
\ln(x + \sqrt{x^2 - 1}) + C^{ste} & = \int \frac{dx}{\sqrt{x^2 - 1} } dx  
\end{align*}
Or, nous avons déjà vu en~\ref{argsinh}, page~\pageref{argsinh} que la dérivée de la fonction  $x \mapsto \arg\mspace{-1mu}\sinh(x)$  vaut $ \frac{1}{ \sqrt{x^2 - 1}}$ ce qui implique que $\ln(x + \sqrt{x^2 - 1}) + C^{ste} = \arg\mspace{-1mu}\sinh\nolimits(x) $. Montrons-le !










\subsection{Avons nous $\arg\mspace{-1mu}\cosh\nolimits(x) = \ln(x + \sqrt{x^2 - 1}) $ ?}

Posons $ y = \arg\mspace{-1mu}\cosh\nolimits(x) $, comme $ \arg\mspace{-1mu}\cosh\nolimits $ est la fonction inverse de $\cosh$, on a $\cosh(y) = \frac{\exp(y) + \exp(-y)}{2} = x$ d'où $2x = \exp(y) + \exp(-y) $ et en multipliant par $\exp(y)$, on obtient l'équation du second degré en $\exp(y)$,
\begin{equation}
\left(\exp(y) \right)^2 - 2x \exp(y) + 1 = 0, 
\end{equation}
dont le discriminant $\Delta$ vaut $4x^2 - 4 \neq 0$, ainsi les solutions s'écrivent $ \exp(y_1) = x + \frac{\sqrt{4x^2 - 4}}{2} = x + \sqrt{x^2-1}$ et $ \exp(y_2) = x - \frac{\sqrt{4x^2 - 4}}{2} = x - \sqrt{x^2-1}$. 

On ne retient que la solution $ \exp(y) = x + \sqrt{x^2-1}$ puisque la fonction exponentielle est toujours positive et que $ \exp(y_2) = x - \sqrt{x^2-1} < 0 $.

Finalement, $y = \arg\mspace{-1mu}\cosh\nolimits(x) = \ln(x + \sqrt{x^2-1})$.


\begin{center}
\sageplot[width=.3\textwidth]{plot(log(x+sqrt(x^2-1)), x, -5, 5)} 
%\sageplot[width=.3\textwidth]{plot(log(x-sqrt(x^2-1)), x, -5, 0)} 
\sageplot[width=.3\textwidth]{plot(arccosh(x), x, -5, 5)} \\
Les représentations graphiques de $\ln(x + \sqrt{x^2-1})$ et de $\arg\mspace{-1mu}\cosh\nolimits(x)$.
\end{center}


Nous avons montré l'égalité $\arg\mspace{-1mu}\cosh\nolimits(x) = \ln(x + \sqrt{x^2-1}) $.







\end{document}