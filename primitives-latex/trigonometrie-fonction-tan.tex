% !TEX TS-program = sage 
% !TEX TS-program = xelatex
% !TEX encoding = UTF-8 Unicode
% The following lines are standard LaTeX preamble statements.


%\documentclass[a4paper,12pt]{smfart}

%\documentclass[a4paper,11pt]{book}
%\documentclass[a4paper,landscape,20pt]{extarticle} %% landscape ou bien portrait (par défault)

%\documentclass[a4paper,17pt]{extarticle} %% on peut utiliser les tailles: 8pt, 9pt,14pt, 17pt et 20pt

\documentclass[a4paper,12pt]{report}
%\documentclass[a4paper,landscape,17pt]{extreport} %% on peut utiliser les tailles: 8pt, 9pt,14pt, 17pt et 20pt


\usepackage[french]{babel}
\usepackage[T1]{fontenc}
\usepackage[utf8]{inputenc}

%\usepackage{amsmath,amssymb,amsfonts,url,xspace,smfthm}
\usepackage{amsmath,amsthm,amssymb,amsfonts,url,xspace}
\usepackage{mathtools}
\usepackage[mathscr]{eucal}
%\usepackage{exscale}

\usepackage{diagbox}

\usepackage{fancybox}

\usepackage{xcolor}

%%%%%%%%%%%%%%%%%%%%%%
% - les theoremes
\newtheorem{definition}{D\'efinition}
\newtheorem{theoreme}{Th\'eor\`eme}
\newtheorem{proposition}{Proposition}
\newtheorem{lemme}{Lemme}
\newtheorem{corollaire}{Corollaire}
\newtheorem{propriete}{Propri\'et\'e}
\newtheorem{axiome}{Axiome}
\newcommand{\initialisecompteurs}{
       \setcounter{definition}{0}%
       \setcounter{theoreme}{0}%
       \setcounter{proposition}{0}%
       \setcounter{lemme}{0}%
       \setcounter{corollaire}{0}%
       \setcounter{propriete}{0}%
       \setcounter{axiome}{0}%
       \setcounter{section}{0}%
       \setcounter{subsection}{0}%
       \setcounter{subsubsection}{0}%
}

\usepackage[left=1.5cm, right=1.5cm, top=1.5cm, bottom=2.cm, bindingoffset=0.cm, headsep=0.cm, headheight=0.cm]{geometry}

%\usepackage{qrcode}
%\usepackage[draft,forget]{qrcode}
%\usepackage{rmsmacros} 
\usepackage{graphicx}
%\usepackage{hyperref}
%\usepackage[hidelinks]{hyperref}
%\usepackage{tabvar}
\usepackage[tight]{shorttoc}
\newcommand{\sommaire}{\shorttoc{Sommaire}{1}}

\graphicspath{ {./images/} }
\everymath{\displaystyle}

%%%%%%%%%%%%%%%%%%%%%%%%%%%%%%%%

%\usepackage[boldsans]{ccfonts}

%\usepackage{concmath}

\usepackage{cmbright}

%\usepackage[euler-digits]{eulervm} 


%\usepackage{beton,euler}

%%%%%%%%%%%%%%%%%%%%%%%%%%%%%%%%


\def\eclaire{\mathbb}

\def\R{\ensuremath{\eclaire R}}
\def\N{\ensuremath{\eclaire N}}
\def\C{\ensuremath{\eclaire C}}
\def\Z{\ensuremath{\eclaire Z}}
\def\Q{\ensuremath{\eclaire Q}}

\def\Id{\ensuremath{\mathrm{Id}}}

%\newcommand{\sh}{\mathop{\mathrm{sh}}\nolimits}
\renewcommand{\sinh}{\mathop{\mathrm{sh}}}
\renewcommand{\cosh}{\mathop{\mathrm{ch}}}
\renewcommand{\tanh}{\mathop{\mathrm{th}}}

\renewcommand{\arg}{\mathop{\mathrm{Arg}}}

\renewcommand{\arcsin}{\mathop{\mathrm{Arc\mspace{2mu}sin}}}
\renewcommand{\arccos}{\mathop{\mathrm{Arc\mspace{2mu}cos}}}
\renewcommand{\arctan}{\mathop{\mathrm{Arc\mspace{2mu}tan}}}
\renewcommand{\Im}{\mathop{\mathfrak{I}\mathrm{m}}}
\renewcommand{\Re}{\mathop{\mathfrak{R}\mathrm{m}}}


%\DeclareMathOperator{\sh}{sh}
%\DeclareMathOperator{\ch}{ch}


\renewcommand{\baselinestretch}{1.5}


% Only one command is required to use Sage within the LaTeX source:
\usepackage{sagetex}


%%%%%%%%%%%%%%%%%%%%%%%%%%%%%%%%

\newcommand{\limh}[1]{\lim\limits_{h\to 0} \frac{#1(x+h)-#1(x)}{h}}

%%%%%%%%%%%%%%%%%%%%%%%%%%%%%%%%




\begin{document} 

%\pagecolor{gray!45}

\setlength{\parindent}{0pt}


%\bf \Huge

\title{Les primitives et les dérivées.}
%\date{Ce document est destiné à un affichage sur un écran et non pas à une impression sur papier.}


\author{Louis Herzog} 

\maketitle 

\setcounter{tocdepth}{2}
\tableofcontents
%\sommaire





\section{La fonction  $x \mapsto \tan(x)$.}
Définissons nos fonctions dans {\texttt{Sage}}
\begin{sageblock}
    f(x) = tan(x)
    g(x) = diff(f(x),x)
    F(x) = integrate(f(x),x)
\end{sageblock}

\begin{center}
\sageplot[width=.3\textwidth]{plot(tan(x), x, -1.4, 1.4)} \\
La représentation graphique de $x\mapsto \tan(x)$ sur l'intervalle ouvert $ \left] -\frac{\pi}{ 2} , \frac{\pi}{ 2} \right[ $.
\end{center}

La fonction $x \mapsto \tan(x)$ étant périodique de période $\pi$, on choisit de restreindre le domaine de définition à l'intervalle ouvert $ \left] -\frac{\pi}{ 2} , \frac{\pi}{ 2} \right[ $.

\subsection{Dérivée de la fonction $x \mapsto \tan(x)$.}
\begin{align*}
\tan(x)' 
& =  \left(\frac{\sin(x)}{\cos(x)}\right)' \\ 
& =  \frac{\cos(x) \times \cos(x) - (-\sin(x)) \times \sin(x)}{\cos^2(x)}  \\ 
& =  \frac{\cos(x) \times \cos(x) + \sin(x)\times \sin(x)}{\cos^2(x)}  \\ 
& =  \frac{1}{\cos^2(x)} = 1 + \tan^2(x)
\end{align*}
\paragraph{On vérifie ce résultat avec Sage.}
La dérivée de $\sage{f(x)} = \sage{g(x)} $.


\subsection{Calcul d'une primitive de la fonction  $x \mapsto \tan(x)$.}
On a $\tan(x)=\frac{\sin(x)}{\cos(x)}$, alors $\int \tan(x) \, dx =\int \frac{\sin(x)}{\cos(x)}\, dx$.\\
Je pose $u(x)=\cos(x)$ donc $u'(x)= -\sin(x) \,dx$ et par ce changement de variable on a $\int \tan(x) \, dx = \int \frac{\sin(x)}{\cos(x)}\, dx  = -\int \frac{u'}{u} = -\ln|u|  = \ln\left(\frac{1}{|u|}\right)  = \ln\left(\frac{1}{|\cos(x)|}\right) +C^{ste}$.

Or, on a choisi le domaine de définition de la fonction $x \mapsto \tan(x)$ restreint à l'intervalle ouvert $ \left] -\frac{\pi}{ 2} , \frac{\pi}{ 2} \right[ $, par conséquent $\cos(x)$ est positif sur cet intervalle donc $|\cos(x)| = \cos(x)$.

Finalement, $ \ln\left(\frac{1}{\cos(x)}\right) +C^{ste}$ est une primitive de $x \mapsto \tan(x)$.
\paragraph{On vérifie ce résultat avec Sage.}
Une primitive de $\sage{f(x)}$ est la fonction définie à une constante près $x \mapsto \sage{F(x)} + C^{ste} $.
Sage utilise la fonction $x\mapsto \sec$ qui est la fonction paire $x\mapsto \frac{1}{\cos(x)}$ périodique de période $2\pi$ définie sur $\R-\{ \frac{\pi}{2}+k\pi, k\in\Z\}$. On retrouve bien le résultat précédent.








\subsection{Calcul de la dérivée de la fonction $x \mapsto \arctan(x) $.}
Définissons nos fonctions dans {\texttt{Sage}}
\begin{sageblock}
    f(x) = arctan(x)
    g(x) = diff(f(x),x)
    F(x) = integrate(f(x),x)
\end{sageblock}
Pour ce calcul, il faut utiliser le calcul de la dérivée d'une fonction composée. On a $\tan(\arctan(x))=x$, par conséquent la dérivée de cette expression s'exprime par $ \tan'(\arctan(x)) \times \arctan\nolimits'(x) = 1$, d'où $\arctan\nolimits'(x) = \frac{1}{\tan'(\arctan(x))} $.

La difficulté est maintenant de déterminer $\tan'(\arctan(x))$, or on sait que pour tout $X \in \R$, on a $ \tan'(X) =1+\tan^2(X) $, avec $X= \arctan(x)$ d'où $\tan'(\arctan(x)) = 1+x^2$.

Finalement, $\arctan\nolimits'(x) = \frac{1}{1+x^2}$.
\paragraph{On vérifie ce résultat avec Sage.}
La dérivée de $\sage{f(x)} = \sage{g(x)} $.

\begin{center}
\sageplot[width=.3\textwidth]{plot(f(x), x, -10, 10)} 
\sageplot[width=.3\textwidth]{plot(tan(x), x, -1.4, 1.4)}\\
Les représentations graphiques de $x \mapsto \sage{f(x)} $ et de $x\mapsto \tan(x)$.
\end{center}
On peut maintenant entreprendre le calcul de la primitive de la  fonction  $x \mapsto \arctan(x) $.


\subsection{Calcul de la primitive de la fonction  $x \mapsto \arctan(x) $.}
Je pose que $u(x)$  est égal à la fonction $\arctan(x)$ et $v'(x)$ est égal $dx$  d'où $u'(x)$  est égal à la fonction $ \frac{1}{1+ x^2} $ et $v(x)$ est égal $x$.
Alors on a $\int \arctan(x) \, dx = x \times \arctan(x) -\int \frac{1}{1+x^2} \times x \, dx $.

\paragraph{Calcul de $\int \frac{x}{1+ x^2} \, dx $.}
$\int \frac{x}{1+ x^2} \, dx = \frac{1}{2} \int \frac{d(1+x^2)}{1+ x^2} = \frac{1}{2} \ln \left| 1+ x^2 \right| $.

D'où $\int \arctan(x) \, dx = x \arctan(x) - \frac{1}{2} \ln \left| 1+ x^2 \right| + C^{ste} $. 
Finalement, une primitive de la fonction $x \mapsto \arctan(x) $ est une fonction $x \mapsto x \arctan(x) -\ln\left( \sqrt{1+ x^2}\right) + C^{ste} $ ou encore $x \mapsto x \arctan(x) +\ln\left( \frac{1}{\sqrt{1+ x^2}}\right) + C^{ste} $.
\paragraph{On vérifie ce résultat avec Sage.}
Une primitive de $\sage{f(x)} = \sage{F(x)} + C^{ste}$.













\end{document}