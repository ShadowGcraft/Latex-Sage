% !TEX TS-program = sage 
% !TEX TS-program = xelatex
% !TEX encoding = UTF-8 Unicode
% The following lines are standard LaTeX preamble statements.

%\documentclass[a4paper,12pt]{smfart}

%\documentclass[a4paper,12pt]{smfart}

%\documentclass[a4paper,11pt]{book}
%\documentclass[a4paper,landscape,20pt]{extarticle} %% landscape ou bien portrait (par défault)
\documentclass[a4paper,14pt]{extarticle} %% on peut utiliser les tailles: 8pt, 9pt,14pt, 17pt et 20pt
%\documentclass[a4paper,11pt]{report}
\usepackage[french]{babel}
\usepackage[T1]{fontenc}
\usepackage[utf8]{inputenc}

%\usepackage{amsmath,amssymb,amsfonts,url,xspace,smfthm}
\usepackage{amsmath,amsthm,amssymb,amsfonts,url,xspace}
\usepackage[mathscr]{eucal}

%\usepackage{exscale}

\usepackage{diagbox}

\usepackage{fancybox}

\usepackage{xcolor}

%%%%%%%%%%%%%%%%%%%%%%
% - les theoremes
\newtheorem{definition}{D\'efinition}
\newtheorem{theoreme}{Th\'eor\`eme}
\newtheorem{proposition}{Proposition}
\newtheorem{lemme}{Lemme}
\newtheorem{corollaire}{Corollaire}
\newtheorem{propriete}{Propri\'et\'e}
\newtheorem{axiome}{Axiome}
\newcommand{\initialisecompteurs}{
       \setcounter{definition}{0}%
       \setcounter{theoreme}{0}%
       \setcounter{proposition}{0}%
       \setcounter{lemme}{0}%
       \setcounter{corollaire}{0}%
       \setcounter{propriete}{0}%
       \setcounter{axiome}{0}%
       \setcounter{section}{0}%
       \setcounter{subsection}{0}%
       \setcounter{subsubsection}{0}%
}


\usepackage[left=2.cm,right=2.cm,top=2.cm,bottom=2cm,bindingoffset=0cm, headsep=0cm,headheight=0cm]{geometry}

%\usepackage{qrcode}
%\usepackage[draft,forget]{qrcode}
%\usepackage{rmsmacros} 
\usepackage{graphicx}
%\usepackage{hyperref}
%\usepackage[hidelinks]{hyperref}
%\usepackage{tabvar}
%\usepackage[tight]{shorttoc}
%\newcommand{\sommaire}{\shorttoc{Sommaire}{1}}

\graphicspath{ {./images/} }
\everymath{\displaystyle}
%\usepackage{fontspec}  % fontspec et xunicode sont facultatifs
%\setmainfont[Ligatures=TeX]{Helvetica}   % Police romaine, utilisée dans le corps du document
%\setmainfont[Ligatures=TeX]{Lucida Grande}   % Police romaine, utilisée dans le corps du document



\def\eclaire{\mathbb}

\def\R{\ensuremath{\eclaire R}}
\def\N{\ensuremath{\eclaire N}}
\def\C{\ensuremath{\eclaire C}}
\def\Z{\ensuremath{\eclaire Z}}
\def\Q{\ensuremath{\eclaire Q}}

%\newcommand{\sh}{\mathop{\mathrm{sh}}\nolimits}
\renewcommand{\sinh}{\mathop{\mathrm{sh}}}
\renewcommand{\cosh}{\mathop{\mathrm{ch}}}
\renewcommand{\tanh}{\mathop{\mathrm{th}}}
\renewcommand{\arcsin}{\mathop{\mathrm{Arc\,sin}}}
\renewcommand{\arccos}{\mathop{\mathrm{Arc\,cos}}}
\renewcommand{\arctan}{\mathop{\mathrm{Arc\,tan}}}
\renewcommand{\Im}{\mathop{\mathfrak{I}\mathrm{m}}}
\renewcommand{\Re}{\mathop{\mathfrak{R}\mathrm{m}}}




\DeclareMathOperator{\sh}{sh}
\DeclareMathOperator{\ch}{ch}


\renewcommand{\baselinestretch}{1.5}


% Only one command is required to use Sage within the LaTeX source:
\usepackage{sagetex}

\begin{document} 

\pagecolor{brown!50}

\setlength{\parindent}{0pt}


%\bf \Huge

\section{Quelques considérations préliminaires}

Quelques considérations préliminaires sur le calcul de certaines primitives telles que celles des fonctions 
 $x \mapsto \arccos(x) $,  $x \mapsto \arcsin(x) $ ou bien $x \mapsto \arctan(x) $ une astuce est indispensable. Elle consiste à procéder à une intégration par parties. En effet, n'ayant aucune idée de ces primitives, il est raisonnable de changer leurs rôles respectifs et de considérer, non plus la fonction initiale, mais sa dérivée.
 
 
Le premier objectif est donc de vérifier si celles-ci existent et sont calculables.

\section{La fonction  $x \mapsto \arccos(x) $.}


La restriction de la fonction $x \mapsto \cos(x) $ à l'intervalle $[0,\pi]$ est une bijection de $[0,\pi] \rightarrow [-1,1]$ . Il existe donc une fonction réciproque à la fonction $x \mapsto \cos(x) $ que l'on nomme $x \mapsto \arccos(x) $. C'est également une bijection, elle est continue sur l'intervalle fermé  $ [-1,1]$ et est dérivable sur l'intervalle ouvert $]-1,1[$.

\subsection{Calcul de la dérivée de la fonction $x \mapsto \arccos(x) $.}

Pour ce calcul, il faut utiliser le calcul de la dérivée d'une fonction composée. On a $\cos(\arccos(x)=x)$, par conséquent la dérivée de cette expression s'exprime par $ -\sin(\arccos(x) \times (\arccos(x))' = 1$, d'où $(\arccos(x))' = \frac{-1}{\sin(\arccos(x))} $.

La difficulté est maintenant de déterminer $\sin(\arccos(x))$, or on sait que pour tout $X \in \R$, on a $\sin^2(X) + \cos^2(X) = 1$, d'où $\sin(X) = \sqrt{1-\cos^2(X)}$.

En remplaçant $X$ par $\arccos(x)$, 

on a $\sin(\arccos(x)) = \sqrt{1-\cos^2(\arccos(x))} = \sqrt{1- x^2}$.

Finalement, $(\arccos(x))' = \frac{-1}{\sin(\arccos(x))} =  \frac{-1}{\sqrt{1- x^2}} $.

\subsubsection*{Vérification avec Sage}

\begin{sageblock}
    f(x) = acos(x)
    g(x) = diff(f(x),x,1)
\end{sageblock}

La dérivée de $\sage{f(x)} = \sage{g(x)} $.

%\sageplot[angle=45, width=.5\textwidth][png]{plot1}

\begin{center}
\sageplot[width=.5\textwidth]{plot(f(x), x, -1, 1)} \\
%\sageplot{plot(g(x), x, -1, 1)}
\end{center}


%\sageplot[angle=45, width=.5\textwidth]{plot(f(x), x, -1, 1)} \\
%\sageplot{plot(g(x), x, -1, 1)}


On peut maintenant entreprendre le calcul de la primitive de la  fonction  $x \mapsto \arccos(x) $.

\subsection{Calcul de la primitive de la fonction  $x \mapsto \arccos(x) $.}




Je pose que $u(x)$  est égal à la fonction $\arccos(x)$ et $v'(x)$ est égal $1$  d'où $u'(x)$  est égal à la fonction $ \frac{-1}{\sqrt{1- x^2}} $ et $v(x)$ est égal $x$.

Alors on a $\int \arccos(x) \, dx = x \times \arccos(x) -\int \frac{-1}{\sqrt{1- x^2}} \times x \, dx $.


\subsubsection*{Calcul de $\int \frac{x}{\sqrt{1- x^2}} \, dx $.}

$\int \frac{x}{\sqrt{1- x^2}} \, dx = \frac{-1}{2} \int \frac{d(1-x^2)}{\sqrt{1- x^2}}= -\sqrt{1- x^2} $.


Finalement $\int \arccos(x) \, dx = x \times \arccos(x) - \sqrt{1- x^2} + Cste $

\subsubsection*{Vérification avec Sage}

\begin{sageblock}
    f(x) = acos(x)
    F(x) = integrate(f(x),x)
\end{sageblock}

La primitive de $\sage{f(x)} = \sage{F(x)} $.

\begin{center}
\sageplot[width=.5\textwidth]{plot(F(x), x, 0, pi)}
\end{center}


\section{La fonction  $x \mapsto \arcsin(x) $.}

\subsection{Calcul de la dérivée de la fonction $x \mapsto \arcsin(x) $.}

\subsection{Calcul de la primitive de la fonction  $x \mapsto \arcsin(x) $.}







\section{La fonction  $x \mapsto \arctan(x) $.}

\subsection{Calcul de la dérivée de la fonction $x \mapsto \arctan(x) $.}


\subsection{Calcul de la primitive de la fonction  $x \mapsto \arctan(x) $.}


%Je pose que $u(x)$  est égal à la fonction $\arctan(x)$ et $v'(x)$ est égal $1$ et le calcul devient accessible lorsque j'aurai la connaissance de la dérivée  $\arctan(x)$ et du calcul d'une primitive de la fonction $ x \mapsto \frac{x}{1+x^2}$ 
\section{Essais Sage}

\begin{sagecommandline}
  sage: factor(x^2 + 2*x + 1)
#  (x + 999)^2
\end{sagecommandline}

\begin{sagecommandline}
sage: h = cos(x)^6 + sin(x)^6 + 3 * sin(x)^2 * cos(x)^2; h
sin(x)^6 + cos(x)^6 + 3*sin(x)^2*cos(x)^2
La même après avoir configuré les affichages en latex:

#sage: %display latex
sage: h
Simplifions-la:

sage: h.simplify_trig()
1
\end{sagecommandline}



\end{document}